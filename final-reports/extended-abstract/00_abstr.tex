%%%%%%%%%%%%%%%%%%%%%%%%%%%%%%%%%%%%%%%%%%%%%%%%%%%%%%%%%%%%%%%%%%%%%%
%     File: ExtendedAbstract_abstr.tex                               %
%     Tex Master: ExtendedAbstract.tex                               %
%                                                                    %
%     Author: Andre Calado Marta                                     %
%     Last modified : 2 Dez 2011                                     %
%%%%%%%%%%%%%%%%%%%%%%%%%%%%%%%%%%%%%%%%%%%%%%%%%%%%%%%%%%%%%%%%%%%%%%
% The abstract of should have less than 500 words.
% The keywords should be typed here (three to five keywords).
%%%%%%%%%%%%%%%%%%%%%%%%%%%%%%%%%%%%%%%%%%%%%%%%%%%%%%%%%%%%%%%%%%%%%%

%%
%% Abstract
%%
\begin{abstract}
The building sector currently presents one of the largest economic and environmental footprints. Optimization can minimize this impact by finding more efficient building variants, prior to their construction. Building performance optimization spurs the exploration of (1) algorithmic approaches, to generate multiple design variants, (2) simulation tools, to evaluate design's performance, and (3) optimization algorithms, to seek more efficient design variants. Despite the existence of several optimization algorithms, their application to architectural optimization problems remains infrequent and, generally, only the simplest available algorithm is adopted. Unfortunately, because design evaluation requires time-intensive simulations, this rarely is the most efficient option for addressing a specific problem, frequently leading to unacceptable optimization times and less efficient designs. 

This dissertation evaluates and identifies algorithms tailored for addressing simulation-based optimization problems. In particular, we develop and assess an optimization framework in the context of three architectural case studies involving single- and multi-objective optimization of simulation-based lighting, structural, and cost aspects of buildings. Results show that selecting the adequate algorithm improves the solution's quality and/or the execution time, thus suggesting that different algorithms should be tested before settling for the simplest available one. Finally, this dissertation shows the benefits of optimization to reduce the impact of the building sector and motivates its introduction as an indispensable phase of the architectural design process.
\\
%%
%% Keywords (max 5)
%%
\noindent{{\bf Keywords:}} Derivative-Free Optimization; Single-Objective Optimization; Multi-Objective Optimization; Algorithmic Optimization; Algorithmic Design \\

\end{abstract}

