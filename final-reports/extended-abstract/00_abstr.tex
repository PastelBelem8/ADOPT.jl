%%%%%%%%%%%%%%%%%%%%%%%%%%%%%%%%%%%%%%%%%%%%%%%%%%%%%%%%%%%%%%%%%%%%%%
%     File: ExtendedAbstract_abstr.tex                               %
%     Tex Master: ExtendedAbstract.tex                               %
%                                                                    %
%     Author: Andre Calado Marta                                     %
%     Last modified : 2 Dez 2011                                     %
%%%%%%%%%%%%%%%%%%%%%%%%%%%%%%%%%%%%%%%%%%%%%%%%%%%%%%%%%%%%%%%%%%%%%%
% The abstract of should have less than 500 words.
% The keywords should be typed here (three to five keywords).
%%%%%%%%%%%%%%%%%%%%%%%%%%%%%%%%%%%%%%%%%%%%%%%%%%%%%%%%%%%%%%%%%%%%%%

%%
%% Abstract
%%
\begin{abstract}

The building sector currently presents one of the largest economic and environmental footprints. Optimization can minimize this impact by finding more efficient building variants, prior to their construction. Building performance optimization spurs the exploration of (1) algorithmic approaches, to generate multiple design variants, (2) simulation tools, to evaluate design's performance regarding distinct aspects (e.g., acoustics, thermal, structural, costs), and (3) optimization algorithms, to seek more efficient design variants. Unfortunately, despite the existence of several optimization algorithms, their application to architectural optimization problems is not well-studied and the lack of experience/knowledge often drives architects towards the application of the simplest available algorithm. However, this rarely is the most efficient option for addressing a specific problem, in particular, due to the time-intensive simulations required to evaluate designs. As a result, a poor algorithm selection might lead to unacceptable optimization times and less efficient designs. 

This dissertation evaluates and proposes optimization algorithms specifically tailored for addressing simulation-based optimization problems. In particular, we develop and assess an optimization framework in the context of three architectural case studies involving single- and multi-objective optimization of simulation-based lighting, structural, and cost aspects of buildings. Obtained results reveal that the algorithm that better addresses a specific optimization problem can yield considerable better solutions and/or produce them in considerable less computation time. Moreover, results corroborate the idea that different optimization algorithms should be tested in order to determine the one that better fits the problem to address. Finally, this dissertation shows the benefits of optimization to reduce the impact of the building sector and motivates its introduction as an indispensable phase of the architectural design process.
\\
%%
%% Keywords (max 5)
%%
\noindent{{\bf Keywords:}} Algorithmic Design, Algorithmic Analysis, Algorithmic Optimization, Derivative-Free Optimization, Single-Objective Optimization Multi-Objective Optimization \\

\end{abstract}

