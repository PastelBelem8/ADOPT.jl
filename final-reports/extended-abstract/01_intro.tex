\section{Introduction}
\label{sec:intro}
	
	Since the creation of the digital computer, architecture adopted computer science tools and techniques to change its own practices. This lead to the dissemination of digital modelling tools, which simplified the design of highly complex buildings. However, these days, it is not enough to have a well-designed building, it is also necessary to ensure that it has a good performance at various levels, such as thermal comfort, lighting, structural, and costs, among others. This concern for performance motivated the appearance of a new design approach named \ac{PBD}, which seeks for more efficient design solutions by considering the design's performance, possibly measured by computational tools \cite{Oxman2006PBD}. 
	
	Additionally, the development of several simulation-based analysis tools enabled architects to estimate the design's performance regarding the intended criteria, prior to their construction. These analysis tools take a specialized model of a design (designated analytical model) and simulate their behavior. Through this process, known as \ac{BPS},
	designers can validate whether their design's performance satisfies the efficiency requirements and, ultimately, optimize the design by iteratively remodeling the geometry in order to obtain variations, assessing their performance, and selecting the better ones. Albeit still being very primitive, architects now have the elementary mechanisms required for optimizing their building's designs, which spurs a new \ac{PBD} approach: \ac{BPO}.
	
	\ac{BPO} treats the results produced by simulation tools as the functions to optimize and, consequently, provides an estimate of a design's performance even when analytical solutions are difficult or impossible to derive \cite{Kolda2003}. In these cases, the function to optimize (called objective function) is derived from the simulations' results and its domain corresponds to the range of acceptable designs specified by the architects.
	
	However, \ac{BPO} requires the evaluation of different design variations, which if done manually, implies spending a large amount of time with the application of changes to the design. Moreover, even though digital modeling tools facilitated design changes, when compared to paper-based approaches, they still present obstacles when modeling complex geometry \cite{Ferreira2015GD}.
	
	% Intro and Definition of AD
	To overcome the aforementioned difficulties, the implementation of changes to a design should be simplified, and, in fact, one way of speeding up design changes is to use parametric models. These models have parameters representing degrees of freedom in the design, which the architect is willing to manipulate in the search for a better performing design. A particularly good approach to produce parametric models is through \ac{AD} \cite{Terzidis2006}. Here, instead of directly creating a 3D model in a digital modeling tool, the architect creates a program that can be executed with different values of its parameters to produce the corresponding 3D models. This considerably improves the implementation of design changes, allowing a broader exploration of the design space.
	
	% Motivation for AA
	Even though \ac{AD} enables the automatic generation of multiple design solutions, the implementation of automatic \ac{BPO} processes still requires the evaluation of these designs. To this end, architects still need to create  the analytical models, which can be very different from the 3D models originally produced by the \ac{AD} tool. Therefore, to evaluate the models produced by the \ac{AD} tool, the architect must either (1) manually generate the analytical model for each variation, requiring a considerable amount of time and effort to create or (2) automatically convert the 3D models to the corresponding analytical models using conversion tools that are often fragile and cause errors or information loss. 
	
	% Intro and Definition of AA
	To overcome the limitations associated with the production of analytical models, one can exploit the idea of using \ac{AD} to automatically generate analytical models from the 3D model's algorithmic description. \ac{AA} is an extension of the \ac{AD} approach that automates the (1) generation of analytical models from a design's algorithmic description, (2) setup of the analyisis tool, and (3) the collection of its results \cite{Aguiar2017}. Using this approach, the architect creates the \ac{AD} model reflecting his design's intents and then sets a few configuration parameters according to the analysis tool to be used. % For example, for lighting analysis tools, a truss might be represented by its surfaces, while for structural analysis tools, it might be represented by a graph of nodes and edges. In either case, the same algorithm is capable of generating different models for different types of analysis.
	
	The \ac{AA} approach is important for the automation of \ac{BPO} processes, as it abstracts the production of the analytical model, removing the need for direct human intervention, while reducing the occurrence of errors. Additionally, when combined with \ac{AD}, it provides the required mechanisms to quickly update a design, to generate the corresponding analytical model, to automatically evaluate the design in an analytical tool, and, finally, to collect the results and use them to guide the search for optimal solutions. 

	The emergence of \ac{AD} approaches together with the growing consciousness of both the benefits and limitations of optimizing building designs, led to the development of ready-to-use optimization toolsets (e.g., Galapagos and Opossum). 
		
	Nevertheless, optimization is still rarely sought within the architectural practice, mainly due to the associated limitations: (1) the difficulties in defining the optimization problem; (2) the need for high expertise to select and fine-tune the optimization algorithms; (3) the complexity of the existing optimization tools; and (4) the long computation time of the optimization process \cite{Attia2013,Nguyen2014}. Moreover, the application of optimization algorithms to architectural optimization problems is still little explored and often leads \ac{BPO} practitioners towards the application of the simplest available algorithm, which rarely is the most efficient option. In particular, due to the time-intensive simulations required to evaluate building designs, the improper choice of algorithm might lead to unacceptable optimization times and less efficient designs. 

	The main goal of this dissertation is to identify the most efficient optimization algorithms for the different performance optimization problems that occur in building design and that frequently involve computationally heavy evaluation functions. Moreover, this dissertation aims to encapsulate these algorithms in an optimization framework that simplifies their application and, thus, promotes design optimization in architecture. Finally, we evaluate the proposed framework in the architectural practice and we study the behavior of different algorithms in various real \ac{BPO} problems, including the optimization of simulation-based lighting, structural, and cost aspects of buildings. 