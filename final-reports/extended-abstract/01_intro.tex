\section{Introduction}
\label{sec:intro}

	Environmental and economic concerns are forcing the architectural practice to carefully balance, in building design, the energy, lighting, structural, and cost factors, among others, seeking for more efficient design solutions, a methodology named Performance-Based Design (PBD). PBD benefits both from Algorithmic Design (AD) methodologies, because these offer architects the ability to explore larger design spaces, and simulations tools, because these allow them to analyze the different design solutions within that space. Finally, in order to find the best performing solutions, optimization techniques are required. 

	The combination of AD, simulation, and optimization yields great potential for architectural practice. Firstly, it allows for the discovery of more efficient design variations. Secondly, it enables more informed design processes, as the architect can visualize and influence the optimization process. Finally, it can uncover unexpected designs that, despite belonging to the design space, were not previously considered.

	Despite its benefits, optimization is rarely sought due to the associated limitations, including (1) the difficulties in defining the optimization problem, (2) the need for high expertise to select and fine-tune the optimization algorithms, (3) the complexity of the existing optimization tools, as well as (4) the long computation time of the optimization process (Attia et al. 2013, Nguyen et al. 2014, Cichocka et al. 2017). Furthermore, the lack of confidence in the optimization process and the idea that it is a completely automated process often leads to the perception of optimization as a restrictive process that hinders the architect’s creativity and expressiveness.

\subsection{Building Performance Optimization}
	
	
\subsection{Algorithmic Design}

	The implementation of changes to a design should be simplified, and, in fact, one way of speeding up design changes is to use parametric models. These models have parameters representing degrees of freedom in the design, which the architect is willing to manipulate in the search for a better performing design. A particularly good approach to produce parametric models is through Algorithmic Design \cite{Terzidis2006}. Here, instead of directly creating a 3D model in a \ac{CAD} or \ac{BIM} tool, the architect creates a program that can be executed with different values of its parameters to produce the corresponding 3D models. This considerably improves the implementation of design changes, allowing a borader exploration of the design space.

\subsection{Algorithmic Analysis}
	Similarly to AD, in Algorithmic Analysis (AA), the analytical model is produced by an algorithm, instead of being created by hand or relying on fragile conversion tools (Leitão et al 2017). The novelty is that this algorithm is the same that is used for AD, but configured differently, so as to match the requirements of the analysis tools being used. For example, for lighting analysis tools, a truss might be represented by its surfaces, while for structural analysiss tools, it might be represented by a graph of nodes and edges. In either case, the same algorithm is capable of generating different models for different types of analysis.

	Moreover, AA is also concerned with the automation of the whole analysis process. In practice, this means that not only is the generation of the analytical model automated, but so is the setup of the analysis tool and the collection of its results.

\subsection{Algorithmic Optimization}
	With the ability to quickly update a design, to generate the corresponding analytical model, and, finally, to automatically evaluate the design in an analytical tool, it becomes possible to implement automated optimization processes. We name these processes Algorithmic Optimization.

	AO treats the analyses' results for different variations of the design as the functions to optimize, i.e., as the objective functions. These functions have a domain which corresponds to the range of acceptable designs as specified by the architect. Moreover, since we do not know their mathematical form, thes
	
