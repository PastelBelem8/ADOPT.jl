\section{Algorithmic Optimization in Architecture}
\label{sec: Methodology}

With the ability to quickly update a design, to generate the corresponding analytical model, and, finally, to automatically evaluate the design in an analytical tool, it becomes possible to implement automated optimization processes. We name these processes \ac{AO}.

\ac{AO} treats the analyses' results for different variations of the design as the functions to optimize, i.e., as the objective functions. These functions have a domain which corresponds to the range of acceptable designs as specified by the architect. Moreover, since we do not know their mathematical form, these objective functions are often treated as black- boxes, and, as a result, information about the their derivatives cannot be extracted. For this reason, methods depending on function derivatives cannot be used to address this problem. Instead, we use derivative-free (or black-box) algorithms, which can be divided into three distinct subclasses: direct-search, metaheuristics, and model-based \cite{Conn2009,Glover2003Metaheuristics,Koziel2011}.

\todo{TRABALHAR NA TRANSIÇÃO}
