\section{Algorithmic Optimization in Architecture} % Framework Description
\label{sec: Methodology}
The architectural design paradigm has recently adopted \ac{AD} and \ac{AA} to quickly (1) update a design, (2) generate the corresponding analytical model, and (3) automatically evaluate the design in an analytical tool. These mechanisms laid down the foundations for automated optimization processes, which we named \ac{AO}. This prompted the development of a few optimization architectural tools (e.g., Galapagos, Opossum) implemented on top of visual \ac{AD} tools (e.g., Grasshopper). 

Despite promoting the practice of optimization within architecture, most of them are mainly focused on \ac{SOO}, thus forcing architects to adopt simplified strategies when addressing \ac{MOO} problems. Additionally, the majority of these tools only provide metaheuristics algorithms, which rarely are the best choice for most problems involving expensive objective functions \cite{Wortmann2017GABESTCHOICE}, often incurring in large computational times. Finally, because each tool provides a narrow set of algorithms, every time architects decide to test multiple algorithms, they must spend additional time and efforts to adapt their program and configure other optimization tools.

To overcome these limitations, we propose an optimization framework containing (1) an easy-to-use and intuitive interface to define and solve optimization problems, (2) a wide variety of sampling algorithms and derivative-free optimization algorithms, which currently include $4$ sampling algorithms, $5$ direct-search, $17$ metaheuristics, and $10$ model-based algorithms, (3) a set of performance indicators to measure the quality of different algorithms, and (4) a set of visual and post-processing mechanisms to aid in the interpretation of the optimization results. 

To support the \ac{AO} approach, the optimization framework is combined with an \ac{AD} tool to facilitate the resolution of \ac{BPO} problems. To this end, architects are required to: (1) create the \ac{AD} model reflecting their design intents; (2) select the performance aspects to optimize and, thus, the analysis tools to be used (e.g., thermal, structural, costs), and, finally, (3) select the parameters of the optimization process (e.g., algorithm and algorithm's parameters).

One important aspect that results from this combination is that architects are able to opt for the most suitable optimization algorithm for a specific problem. However, since the best choice is not always known, the optimization framework also provides automated testing mechanisms to enable the sequential execution of multiple optimization algorithms for a specified amount of evaluations. This feature promotes more informed decisions towards an adequate selection of optimization algorithms~\cite{Wortmann2016BBO}.

The interactive visualization of the evaluated design solutions is of great importance for architects, as it allows them to explore and corroborate the optimization results during the optimization process. To this end, the framework provides visualization mechanisms that promote a better comprehension of the optimization process, allowing early error detection and the potential reduction in the overall optimization time by providing the architect with enough information to stop the optimization process sooner. 

