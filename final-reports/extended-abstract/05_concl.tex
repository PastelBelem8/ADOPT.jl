\section{Conclusions}
\label{sec:concl}

Nowadays, with the threat of climate change, resource depletion, and worldwide urbanization, it is not enough to construct well-designed buildings, it is also necessary to optimize them \cite{Wortmann2015AdvSBO}. Architectural practices have, therefore, grown to incorporate considerations about the building's performance in various aspects. The development of computational simulation tools empowered designers with the ability to simulate and estimate a building’s performance. The emergence of these tools and the raising concerns about the environmental and economic impact of buildings led to the development of new design approaches, such as \ac{PBD}, which seek more efficient design solutions by considering the designs’ performance. Taking \ac{PBD} a step further, optimization has unveiled a new performance-based approach called \ac{BPO}. 

Unfortunately, traditional \ac{BPO} methodologies require the evaluation of different design variations, which, in turn, implies spending a large amount of time with the manual application of changes to the design and often leading to difficulties when modeling complex geometry. Moreover, in order to evaluate a design's performance, the corresponding analytical models must be produced, which also comprises a time-consuming and tiresome task. The emergence of algorithmic-based paradigms, like \ac{AD} and \ac{AA}, enabled the implementation of automated optimization processes, as they allow architects to generate multiple design variants with little effort, to automatically produce the corresponding analytical models and to automatically evaluate their performance. 

Optimization algorithms can be coupled with the previously mentioned algorithmic approaches and simulation tools to more efficiently seek for optimal design solutions. Given that a single evaluation may take a considerable amount of time to complete, in order to speed up the optimization process, it becomes necessary to identify the optimization algorithms capable of handling the computationally complex problems that characterize \ac{BPO}, and to devise strategies for its efficient application in architecture. Often disregarded, the selection of the appropriate optimization algorithm can have a significant impact in the overall efficiency of optimization processes and on the quality of the results \cite{Wolpert1997NFLT}.

However, most \ac{BPO} practitioners adopt the simplest available algorithm, such as \acp{EA} \cite{Evins2013}, which typically require hundreds or thousands of evaluations - an infeasible scenario for most \ac{BPO} problems. Conversely, direct-search and model-based algorithms are more promising, frequently yielding better results in fewer evaluations \cite{Waibel2018}. Particularly, model-based algorithms can considerably reduce the time spent in optimization \cite{Wortmann2017GABESTCHOICE}, and, in fact, results show that these algorithms exhibit an average reduction of $50\%$ on the total time spent in optimization.

Along these lines, we addressed optimization algorithms specifically tailored for handling simulation-based optimization problems, where the computational effort of simulation often restricts the number of function evaluations to a few dozens or hundreds. We introduced \ac{AO}, an extension to the \ac{AD} and \ac{AA} approaches, that combines an optimization framework with an \ac{AD} tool to address various design optimization problems, including \ac{BPO}. Results show that no single class, subclass, or algorithm excels at every problem, thus corroborating the \acp{NFLT} for optimization \cite{Wolpert1997NFLT}. 

Moreover, even though other tools rely on \ac{GA}-based algorithms, these rarely achieved the best performance in the evaluated case studies. Instead, other categories, such as model-based or direct-search algorithms, yielded better results. Different factors could change the obtained results (e.g., a different configuration of the algorithms or a lucky random step). Therefore, and contrarily to current architectural practices, we conclude that distinct algorithms behave differently according to the problems' characteristics and that architects should first test various algorithms for a small number of evaluations or for a short amount of time. 
Furthermore, we also conclude that while global optimization algorithms are quicker to converge towards optimal solutions when no additional information is known, local algorithms can be quicker if provided with good starting points and, therefore, should be considered when such information is available.

Regarding the evaluation of different \acp{MOO} algorithms, the lack of consensus regarding the appropriate way to measure their quality makes it difficult to quantify the suitability of each algorithm for \ac{MOO} problems. We conclude that algorithms' quality should be measured through the combination of Pareto front plots and multiple \ac{MOO} performance indicators. Particularly, some indicators should provide a measure of the diversity of the nondominated solutions across the solution space, while others should measure the overall accuracy of the results.

To overcome the identified limitations, the proposed optimization framework includes different categories of optimization algorithms and facilitates their application by abstracting them under a common interface, thus promoting automated optimization processes. To further facilitate the selection of the most appropriate algorithm, it also includes mechanisms to effortlessly test multiple algorithms. The framework was evaluated in the context of two \ac{BPO} problems, which demonstrated its ability to solve real architectural problems characterized by computationally complex objective functions.

%Overall, optimization can be very beneficial for architecture. The combination of simulation tools, algorithmic-based approaches, and optimization algorithms enables the automation of optimization processes within architectural practices. Despite its benefits, the incorrect application of these optimization processes can lead to poor results. Moreover, distinct architectural design problems benefit from different optimization algorithms. However, most architectural optimization tools focus on the same subset of algorithms, which are rarely the most adequate. The proposed framework circumvents these limitations by presenting several optimization algorithms with different characteristics, thus fostering better optimization practices. Finally, the proposed \ac{AO} methodology was shown to benefit the architectural practice, as demonstrated by the case studies.