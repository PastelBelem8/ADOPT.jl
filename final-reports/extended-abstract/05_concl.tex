\section{Conclusions}
\label{sec:concl}

Nowadays, with the threat of climate change, resource depletion, and worldwide urbanization, it is not enough to construct well-designed buildings, it is also necessary to optimize them \cite{Wortmann2015AdvSBO}. Architectural practices have, therefore, grown to incorporate considerations about the building's performance in various aspects. The development of computational simulation tools empowered designers with the ability to simulate and estimate a building’s performance. The emergence of these tools and the raising concerns about the environmental and economic impact of buildings led to the development of new design approaches, such as \ac{PBD}, which seek more efficient design solutions by considering the designs’ performance. Taking design’s performance a step further, optimization has unveilled a new performance-based approach called \ac{BPO}. 

Unfortunately, traditional \ac{BPO} methodologies require the evaluation of different design variations, which, in turn, implies spending a large amount of time with the manual application of changes to the design and often leading to difficulties when manually modeling complex geometry. Moreover, in order to evaluate a design's performance, the corresponding analytical models must be produced, which also comprises a time-consuming and tiresome task. The emergence of algorithmic-based paradigms, like \ac{AD} and \ac{AA}, enabled the implementation of automated optimization processes, as they allow architects to generate multiple design variants with little effort, to automatically produce the corresponding analytical models, and to automatically evaluate their performance. Notwithstanding the benefits attained with algorithmic approaches, the exploration of more efficient designs still comprises a tiresome and time-consuming task. To overcome this limitation, one can exploit optimization algorithms to more efficiently seek for optimal (or near optimal) design solutions.

Notwithstanding the automation of optimization processes, most \ac{BPO} problems require expensive simulation-based objective functions, for which a single evaluation may take a considerable amount of time to complete. In order to speed up the optimization process, it becomes necessary to identify different optimization algorithms capable of handling the computationally complex problems that characterize \ac{BPO}, and to devise strategies for its efficient application in architecture. Often disregarded, the selection of the appropriate optimization algorithm might have a significant impact in the overall efficiency of optimization processes, and also on the quality of the results \cite{Wolpert1997NFLT}.

Despite the benefits associated with a more targeted selection of optimization algorithms, most \ac{BPO} practitioners tend to adopt \acp{EA} \cite{Evins2013, Nguyen2014}, which typically require several hundreds or thousands of evaluations - an infeasible scenario for most \ac{BPO} problems. Conversely, direct-search methods and model-based algorithms are more promising, frequently yielding better results in fewer evaluations \cite{Waibel2018}. Particularly, model-based algorithms can considerably reduce the time spent in optimization \cite{Wortmann2017GABESTCHOICE}. Obtained results show an average reduction by about $50\%$ on the total time spent in optimization.

Along these lines, we addressed optimization algorithms specifically tailored for handling simulation-based optimization problems, where the computational effort associated to each simulation often restricts number of function evaluations to a few hundreds. We introduced an extension to the \ac{AD} and \ac{AA} approaches, called \ac{AO}, which combines an optimization framework with an \ac{AD} tool to allow users to address various design optimization problems, including \ac{BPO}. Results show that no single class, subclass, or algorithm excels at every problem, thus corroborating the \acp{NFLT} for optimization \cite{Wolpert1997NFLT}. 

Most tools and, in particular, \ac{MOO} tools, still rely on evolutionary-based algorithms. However, in the evaluated case studies, these algorithms rarely achieved the best performance. Instead, other categories, such as global model-based or global direct-search algorithms, yielded better results. Different factors could change the obtained results (e.g., a different configuration for the algorithms or a lucky random step). Therefore, and contrarily to current architectural practices, we conclude that distinct algorithms behave differently according to the problems' characteristics and that users should test various algorithms for a small number of evaluations or for a short amount of time. 

Furthermore, we conclude that while global optimization algorithms are quicker to converge towards optimal solutions when no additional information is known, local algorithms can be quicker if provided with good starting points and, therefore, should be considered as potential candidates when such information is available.

Regarding the evaluation of different \acp{MOOA}, the lack of consensus regarding the more appropriate way to measure their quality makes it difficult to quantify the suitability of each algorithm for \ac{MOO} problems. In this study, we conclude that algorithms' quality should be measured through the combination of Pareto front plots and multiple \ac{MOO} performance indicators. Particularly, some indicators should provide a measure of the diversity of the nondominated solutions across the solution space, while others should measure the overall accuracy and convergence of the results.

Based on the conclusions regarding the algorithms' performance, we believe architects should use several optimization algorithms in their workflow and use the optimization approach that better fits their needs and expertise. This conclusion contradicts the approach taken in some architectural optimization plug-ins, which are often limited to a unique optimization approach and to a small subset of algorithms, namely, the \acp{EA}. Unfortunately, as demonstrated, \acp{EA} are not the best choice for \ac{BPO}. To bypass this limitation, the optimization framework proposed in this dissertation includes different categories of optimization algorithms and facilitates their application by abstracting them under a common interface, thus promoting automated optimization processes. To further facilitate the selection of the most appropriate algorithm, it also includes mechanisms to test multiple optimization algorithms effortlessly. The suitability and capabilities of the framework were evaluated in the context of three \ac{BPO} problems, which demonstrated its ability to solve real architectural problems characterized by computationally complex objective functions.

Overall, optimization can be very beneficial for architecture. The combination of simulation tools, algorithmic-based approaches, and optimization algorithms enables the automation of optimization processes within architectural practices. Despite its benefits, the incorrect application of these optimization processes can lead to poor results. Moreover, distinct architectural design problems benefit most from different optimization algorithms, capable of handling them more efficiently. However, most architectural optimization tools focus on the same subset of algorithms, which are rarely the most adequate algorithms. The proposed framework circumvents these limitations by presenting several optimization algorithms with different characteristics, thus fostering better optimization practices. Finally, the proposed \ac{AO} methodology was shown to benefit the architectural practice, as was demonstrated by the three \ac{BPO} case studies.