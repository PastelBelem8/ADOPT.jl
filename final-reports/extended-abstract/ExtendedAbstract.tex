%%%%%%%%%%%%%%%%%%%%%%%% ExtendedAbstract.tex %%%%%%%%%%%%%%%%%%%%%%%%
%                                                                    %
%  Template for the 10-page extended abstract to be submitted for    %
%  the MSc degree conferral at Instituto Superior Tecnico.           %
%                                                                    %
%  Author:                                                           %
%                                                                    %
%       Andre C. Marta                                               %
%       Area Cientifica de Mecanica Aplicada e Aeroespacial          %
%       Departamento de Engenharia Mecanica                          %
%       Instituto Superior Tecnico                                   %
%       Av. Rovisco Pais                                             %
%       1049-001 Lisboa                                              %
%       Portugal                                                     %
%       Tel: +351 21 841 9466                                        %
%                        3466 (extension)                            %
%       Email: andre.marta@ist.utl.pt                                %
%                                                                    %
%  Created:       Dec  2, 2011                                       %
%  Last Modified: Dec 27, 2011                                       %
%%%%%%%%%%%%%%%%%%%%%%%%%%%%%%%%%%%%%%%%%%%%%%%%%%%%%%%%%%%%%%%%%%%%%%
% This document uses the LaTeX class file "article.cls"              %
%%%%%%%%%%%%%%%%%%%%%%%%%%%%%%%%%%%%%%%%%%%%%%%%%%%%%%%%%%%%%%%%%%%%%%
\documentclass[8pt,a4paper,twocolumn]{article}

%%%%%%%%%%%%%%%%%%%%%%%%%%%%%%%%%%%%%%%%%%%%%%%%%%%%%%%%%%%%%%%%%%%%%%
% Document preamble
%%%%%%%%%%%%%%%%%%%%%%%%%%%%%%%%%%%%%%%%%%%%%%%%%%%%%%%%%%%%%%%%%%%%%%
%% Builds upon the graphics  package, providing a key-value interface
%% for optional arguments to the \includegraphics command that go far
%% beyone what the graphics package offers.
%% http://www.ctan.org/tex-archive/help/Catalogue/entries/graphicx.html
%% if you use PostScript figures in your article
%% use the graphics package for simple commands
%% \usepackage{graphics}
%% or use the graphicx package for more complicated commands
%% \usepackage{graphicx}
%% or use the epsfig package if you prefer to use the old commands
%% \usepackage{epsfig}
\usepackage{graphicx} % Enhanced LaTeX Graphics
\usepackage{siunitx}

%Tipo de letra Arial
\usepackage{helvet}
\renewcommand{\familydefault}{\sfdefault}

% acentos e cedilhas
\usepackage[utf8]{inputenc}
%\usepackage[T1]{fontenc}

% Multiple figures
%\usepackage{subfigure} % subcaptions for subfigures
%\usepackage{subfigmat} % matrices of similar subfigures

\usepackage[font=footnotesize, skip = 1pt, labelfont=bf]{caption}
\usepackage[font=footnotesize]{subcaption}

% Declaring new column types
% 'dcolumn' package defines D to be a column specifier with
% three arguments: D{<sep.tex>}{<sep.dvi>}{<decimal places>}
%                  D{<sep.tex>}{<sep.dvi>}{<left digit places>.<right digit places>}
\usepackage{dcolumn}           % decimal-aligned tabular math columns
% d takes a single argument specifying the number of decimal places, e.g., d{2}
% or the number of digits to the left and right of the seperator, e.g., d{3.2}
\newcolumntype{.}   {D{.}{.}{-1}} % column alignedd on the point separator '.'
\newcolumntype{d}[1]{D{.}{.}{#1}} % column centered on the point separator '.'
\newcolumntype{e}   {D{E}{E}{-1}} % column centered on the exponent 'E'
\newcolumntype{E}[1]{D{E}{E}{#1}} % column centered on the exponent 'E'

%% American Mathematical Society (AMS) plain Tex macros
%%
%% The amsmath package is the principal package in the AMS-LaTeX distribution
%% http://www.ctan.org/tex-archive/help/Catalogue/entries/amsmath.html
\usepackage{amsmath}
\DeclareMathSizes{7}{7}{3}{3} 
\usepackage{pifont}
%%
%% The amsfonts package provides extended TeX fonts
%% http://www.ctan.org/tex-archive/help/Catalogue/entries/amsfonts.html
\usepackage{amsfonts}
%% The amssymb package provides various useful mathematical symbols
\usepackage{amssymb}
%%
%% The amsthm package provides extended theorem environments
%% http://www.ctan.org/tex-archive/help/Catalogue/entries/amsthm.html
\usepackage{amsthm}

%% Improves the interface for defining floating objects such as figures and tables.
%% The package also provides the H float modifier option of the obsolete here package.
%% http://www.ctan.org/tex-archive/help/Catalogue/entries/float.html
\usepackage{float}

%% Control sectional headers. 
%% http://www.ctan.org/tex-archive/help/Catalogue/entries/sectsty.html
\usepackage{sectsty}
%%
%% Redefine the font size of the 'section' and 'subsection' headings
\newcommand{\myFontSize}{\fontsize{9}{0}\selectfont}
\sectionfont{\myFontSize}       % 10pt, Bold face (default)
\subsectionfont{\myFontSize} % 10pt, Plain face

%% Select alternative section titles.
%% http://www.ctan.org/tex-archive/help/Catalogue/entries/titlesec.html
\usepackage{titlesec}
\usepackage{booktabs}
%\usepackage{multirow}
%\usepackage{array}
\usepackage{csquotes}% Recommended
%\usepackage[style=authoryear, backend=bibtex, doi=false,isbn=false,url=false,eprint=false,dashed=false,maxcitenames=2, maxbibnames=100]{biblatex}
%\addbibresource{library.bib}
\usepackage[{english,noabbrev}]{cleveref}
\usepackage[nolist,nohyperlinks]{acronym}
\usepackage[textwidth=2cm, textsize=small]{todonotes}
%%
%% Left indent, before and after spacing
%% (The starred version kills the indentation of the paragraph following the title)
\titlespacing*{\section}{0pt}{10pt}{0pt}
\titlespacing*{\subsection}{0pt}{10pt}{0pt}

%% Section numbers with trailing dots. 
%% http://www.ctan.org/tex-archive/help/Catalogue/entries/secdot.html
\usepackage{secdot}
\usepackage{epstopdf}
%%
%% Also put a dot after the subsection number
\sectiondot{subsection}
%% Set a space between dot and heading text
\sectionpunct{section}{. }    % By default, \sectiondot places a \quad
\sectionpunct{subsection}{. } % after the number

% These are exact settings for a A4 page with top margin of
% 25 mm, bottom margin of 30 mm, left and right margins of 25 mm,
% printable area 242 X 160 mm.

\setlength{\topmargin}{-10.4mm}
\setlength{\headheight}{0.0mm}
\setlength{\headsep}{10.0mm}
\setlength{\textwidth}{160mm}
\setlength{\textheight}{242mm}
\setlength{\oddsidemargin}{0mm}
\setlength{\evensidemargin}{0mm}
\setlength{\marginparwidth}{0mm}
\setlength{\marginparsep}{0mm}

% New command to refer to equations as Eq.(1),Eq.(2),...
\newcommand{\eqnref}[1]{Eq.(\ref{#1})}

%%%%%%%%%%%%%%%%%%%%%%%%%%%%%%%%%%%%%%%%%%%%%%%%%%%%%%%%%%%%%%%%%%%%%%%%%%%%%%%%%%%%%%%%
% Title, authors and addresses

\title{\textbf{Optimization of Time-Consuming Objective Functions} \\
	\Large Derivative-Free Approaches and their Application in Architecture \\
	\vspace{2mm}
	[Extended Abstract]}
\date{May 2019}
\author{Catarina Belém\\ catarina.belem@tecnico.ulisboa.pt \\ Instituto Superior Técnico, Lisboa, Portugal}



%%%%%%%%%%%%%%%%%%%%%%%%%%%%%%%%%%%%%%%%%%%%%%%%%%%%%%%%%%%%%%%%%%%%%%%%%%%%%%%%%%%%%%%%
\begin{document}

\begin{acronym}[H.264/SVC]
	% A	--------------------------------------------------------------
	\acro{AD}{Algorithmic Design}
	\acro{AA}{Algorithmic Analysis}
	\acro{AO}{Algorithmic Optimization}
	\acro{API}{Application Programming Interface}
	\acro{aPF}{approximated Pareto Front}
	
	% B --------------------------------------------------------------
	\acro{BIM}{Building Information Modeling}
	\acro{BOBYQA}{Bound Optimization BY Quadratic Approximation}
	\acro{BPO}{Building Performance Optimization}
	\acro{BPS}{Building Performance Simulation}
	
	% C --------------------------------------------------------------
	\acro{CAD}{Computer-Aided Design}
	\acro{COBYLA}{Constrained Optimization BY Linear Approximation}
	\acro{cPF}{combined Pareto Front}
	
	% D --------------------------------------------------------------
	\acro{DFO}{Derivative-Free Optimization}
	\acro{DMS}{Direct MultiSearch}
	\acro{DIRECT}{DIviding RECTangles}
	
	% E --------------------------------------------------------------
	\acro{EA}{Evolutionary Algorithm}	
	\acro{ER}{Error Ratio}	
	%\acrodefplural{ES}{Evolution Strategies}
	\acro{ES}{Evolution Strategy}		
	
	% G --------------------------------------------------------------
	\acro{GA}{Genetic Algorithm}
	\acro{GD}{Generational Distance}
	\acro{GPR}{Gaussian Process Regressor}
	\acro{GUI}{Graphical User Interface}
	
	% H --------------------------------------------------------------
	\acro{HV}{Hypervolume}
	
	% I --------------------------------------------------------------
	\acro{IGD}{Inverted Generational Distance}
	
	% L --------------------------------------------------------------
	
	% M	--------------------------------------------------------------
	\acro{ML}{Machine Learning}
	\acro{MLP}{Multi-Layer Perceptron}
	\acro{MOEA}{Multi-Objective Evolutionary Algorithm}
	\acro{MPFE}{Maximum Pareto Front Error}
	\acro{MOO}{Multi-Objective Optimization}
	\acro{MOOA}{Multi-Objective Optimization Algorithm}
	
	% N --------------------------------------------------------------
	\acro{NFLT}{No Free Lunch Theorem}		
	\acro{NMS}{Nelder-Mead Simplex}
	\acro{NSGA-II}{Non-dominated Sorting Genetic Algorithm II}
	
	% O --------------------------------------------------------------	
	\acro{ONVG}{Overall Non-dominated Vector Generation}
	\acro{ONVGR}{Overall Non-dominated Vector Generation Ratio}
	
	
	% P --------------------------------------------------------------
	\acro{PBD}{Performance-Based Design}
	\acro{PRAXIS}{Principal Axis}
	\acro{PSO}{Particle-Swarm Optimization}
	
	% R	--------------------------------------------------------------
	\acro{RBF}{Radial Basis Function}	
	\acro{RF}{Random Forest}	
	
	% S	--------------------------------------------------------------
	\acro{sUDI}{spatial Useful Daylight Illumination}
	\acro{SPEA2}{Strength Pareto Evolutionary Algorithm 2}
	\acro{SOO}{Single-Objective Optimization}
	\acro{SVR}{Support Vector Regression}
	% T	--------------------------------------------------------------
	\acro{tPF}{true Pareto Front}
	
\end{acronym}
% Begin one column section for title and abstract
%
% http://www.faqs.org/faqs/de-tex-faq/part5/
\twocolumn[
\begin{@twocolumnfalse}
\maketitle

	%%%%%%%%%%%%%%%%%%%%%%%%%%%%%%%%%%%%%%%%%%%%%%%%%%%%%%%%%%%%%%%%%%%%%%
%     File: ExtendedAbstract_abstr.tex                               %
%     Tex Master: ExtendedAbstract.tex                               %
%                                                                    %
%     Author: Andre Calado Marta                                     %
%     Last modified : 2 Dez 2011                                     %
%%%%%%%%%%%%%%%%%%%%%%%%%%%%%%%%%%%%%%%%%%%%%%%%%%%%%%%%%%%%%%%%%%%%%%
% The abstract of should have less than 500 words.
% The keywords should be typed here (three to five keywords).
%%%%%%%%%%%%%%%%%%%%%%%%%%%%%%%%%%%%%%%%%%%%%%%%%%%%%%%%%%%%%%%%%%%%%%

%%
%% Abstract
%%
\begin{abstract}

Sed ut perspiciatis unde omnis iste natus error sit voluptatem accusantium doloremque laudantium, totam rem aperiam, eaque ipsa quae ab illo inventore veritatis et quasi architecto beatae vitae dicta sunt explicabo. Nemo enim ipsam voluptatem quia voluptas sit aspernatur aut odit aut fugit, sed quia consequuntur magni dolores eos qui ratione voluptatem sequi nesciunt. Neque porro quisquam est, qui dolorem ipsum quia dolor sit amet, consectetur, adipisci velit, sed quia non numquam eius modi tempora incidunt ut labore et dolore magnam aliquam quaerat voluptatem. Ut enim ad minima veniam, quis nostrum exercitationem ullam corporis suscipit laboriosam, nisi ut aliquid ex ea commodi consequatur? Quis autem vel eum iure reprehenderit qui in ea voluptate velit esse quam nihil molestiae consequatur, vel illum qui dolorem eum fugiat quo voluptas nulla pariatur?

Sed ut perspiciatis unde omnis iste natus error sit voluptatem accusantium doloremque laudantium, totam rem aperiam, eaque ipsa quae ab illo inventore veritatis et quasi architecto beatae vitae dicta sunt explicabo. Nemo enim ipsam voluptatem quia voluptas sit aspernatur aut odit aut fugit, sed quia consequuntur magni dolores eos qui ratione voluptatem sequi nesciunt. Neque porro quisquam est, qui dolorem ipsum quia dolor sit amet, consectetur, adipisci velit, sed quia non numquam eius modi tempora incidunt ut labore et dolore magnam aliquam quaerat voluptatem. Ut enim ad minima veniam, quis nostrum exercitationem ullam corporis suscipit laboriosam, nisi ut aliquid ex ea commodi consequatur? Quis autem vel eum iure reprehenderit qui in ea voluptate velit esse quam nihil molestiae consequatur, vel illum qui dolorem eum fugiat quo voluptas nulla pariatur?
\\
%%
%% Keywords (max 5)
%%
\noindent{{\bf Keywords:}} Demo word, Lorem, Ipsum \\

\end{abstract}



\end{@twocolumnfalse}
]
	\section{Introduction}
\label{sec:intro}
	
	Since the creation of the digital computer, architecture adopted computer science tools and techniques to change its own practices. This lead to the dissemination of digital modelling tools, which simplified the design of highly complex buildings. However, these days, it is not enough to have a well-designed building, it is also necessary to ensure that it has a good performance at various levels, such as thermal comfort, lighting, structural, and costs, among others. This concern for performance motivated the appearance of a new design approach named \ac{PBD}, which seeks for more efficient design solutions by considering the design's performance, possibly measured by computational tools \cite{Oxman2006PBD}. 
	
	Additionally, the development of several simulation-based analysis tools enabled architects to estimate the design's performance regarding the intended criteria, prior to their construction. These analysis tools take a specialized model of a design (designated analytical model) and simulate their behavior. Through this process, known as \ac{BPS},
	designers can validate whether their design's performance satisfies the efficiency requirements and, ultimately, optimize the design by iteratively remodeling the geometry in order to obtain variations, assessing their performance, and selecting the better ones. Albeit still being very primitive, architects now have the elementary mechanisms required for optimizing their building's designs, which spurs a new \ac{PBD} approach: \ac{BPO}.
	
	\ac{BPO} treats the results produced by simulation tools as the functions to optimize and, consequently, provides an estimate of a design's performance even when analytical solutions are difficult or impossible to derive \cite{Kolda2003}. In these cases, the function to optimize (called objective function) is derived from the simulations' results and its domain corresponds to the range of acceptable designs specified by the architects.
	
	However, \ac{BPO} requires the evaluation of different design variations, which if done manually, implies spending a large amount of time with the application of changes to the design. Moreover, even though digital modeling tools facilitated design changes, when compared to paper-based approaches, they still present obstacles when modeling complex geometry \cite{Ferreira2015GD}.
	
	% Intro and Definition of AD
	To overcome the aforementioned difficulties, the implementation of changes to a design should be simplified, and, in fact, one way of speeding up design changes is to use parametric models. These models have parameters representing degrees of freedom in the design, which the architect is willing to manipulate in the search for a better performing design. A particularly good approach to produce parametric models is through \ac{AD} \cite{Terzidis2006}. Here, instead of directly creating a 3D model in a digital modeling tool, the architect creates a program that can be executed with different values of its parameters to produce the corresponding 3D models. This considerably improves the implementation of design changes, allowing a broader exploration of the design space.
	
	% Motivation for AA
	Even though \ac{AD} enables the automatic generation of multiple design solutions, the implementation of automatic \ac{BPO} processes requires the evaluation of these designs. To this end, architects need to create the analytical models, which can be very different from the 3D models originally produced by the \ac{AD} tool. Therefore, to evaluate the models produced by the \ac{AD} tool, the architect must either (1) manually generate the analytical model for each variation, requiring a considerable amount of time and effort to create or (2) automatically convert the 3D models to the corresponding analytical models using conversion tools that are often fragile and cause errors or information loss. 
	
	% Intro and Definition of AA
	To overcome the limitations associated with the production of analytical models, one can exploit the idea of using \ac{AD} to automatically generate analytical models from the 3D model's algorithmic description. \ac{AA} is an extension of the \ac{AD} approach that automates the (1) generation of analytical models from a design's algorithmic description, (2) setup of the analyisis tool, and (3) the collection of its results \cite{Aguiar2017}. Using this approach, architects create the \ac{AD} model reflecting their design's intents and then set a few configuration parameters according to the analysis tool to be used. % For example, for lighting analysis tools, a truss might be represented by its surfaces, while for structural analysis tools, it might be represented by a graph of nodes and edges. In either case, the same algorithm is capable of generating different models for different types of analysis.
	
	\ac{AA} is crucial for the automation of \ac{BPO} processes, as it abstracts the production of the analytical model, while removing the need for direct human intervention and reducing the occurrence of errors. Thus, when combined with \ac{AD}, it provides the required mechanisms to quickly update a design, to generate the corresponding analytical model, to automatically evaluate the design in an analytical tool, and, finally, to collect the results and use them to guide the search for optimal solutions. 

	The emergence of \ac{AD} approaches together with the growing consciousness of both the benefits and limitations of optimizing building designs, led to the development of ready-to-use optimization toolsets (e.g., Galapagos and Opossum). 
		
	Nevertheless, optimization is still rarely sought within the architectural practice, mainly due to the associated limitations: (1) the difficulties in defining the optimization problem; (2) the need for high expertise to select and fine-tune the optimization algorithms; (3) the complexity of the existing optimization tools; and (4) the long computation time of the optimization process \cite{Attia2013,Nguyen2014}. Moreover, the application of optimization algorithms to architectural optimization problems is still little explored and often leads \ac{BPO} practitioners towards the application of the simplest available algorithm, which rarely is the most efficient option. In particular, due to the time-intensive simulations required to evaluate building designs, the improper choice of algorithm might lead to unacceptable optimization times and less efficient designs. 

	The main goal of this dissertation is to identify the most efficient optimization algorithms for the different performance optimization problems that occur in building design and that frequently involve computationally heavy evaluation functions. Moreover, this dissertation aims to encapsulate these algorithms in an optimization framework that simplifies their application and, thus, promotes design optimization in architecture. Finally, we evaluate the proposed framework in the architectural practice and we study the behavior of different algorithms in various real \ac{BPO} problems, including the optimization of simulation-based lighting, structural, and cost aspects of buildings. 
	
	\todo{Falta fazer ligaçao p/ optimização}
	% A Theory section should extend, not repeat, the background to the
% article already dealt with in the Introduction and lay the
% foundation for further work.

\section{Optimization Algorithms}
\label{sec:backg}

Optimization algorithms can be classified differently according to their properties. One important classification regards the extent of the search for optimal solutions, which can be global or local. Local optimization algorithms strive to find a locally optimal solution, i.e., for which the objective function yields a better value than for all the other solutions in its vicinity. Moreover, local algorithms are usually highly sensitive to the starting point of the search and they tend to focus on smaller regions of the solution space. In contrast, global optimization algorithms strive to find globally optimal solutions, i.e., the best of all the locally optimal solutions. To that end, these algorithms explore larger regions of the solution space and typically require several evaluations to find global optima. However, the increased number of evaluations can be problematic in problems involving expensive objective functions. To minimize this impact, a good approach might be to apply a global algorithm to identify a promising region and then apply a local algorithm to more rapidly find the optima within that region. 

A second classification differentiates deterministic and stochastic algorithms depending on the determinism of the algorithms' outcomes. Given the same starting point and configuration, deterministic algorithms systematically apply the same sequence of steps and, as a consequence, they always return the same result. In contrast, stochastic algorithms include some form of randomness within their description and, therefore, often yield different results for the same starting point and algorithm's configuration.

The third, and final, distinction is between derivative-based and derivative-free algorithms, which differ in the type of information used during the search. Derivative-based (or gradient-based) algorithms explore information from the derivatives of objective functions, i.e., the direction and magnitude of the greatest increase of the function, to guide the search. % Consequently, they solve problems explicitly defined through mathematical forms very efficiently, as the derivatives' information is easily available. 
However when the objective function's analytical form is unknown and the derivatives are unavailable, these algorithms cannot be applied. Although finite-difference methods could be applied to approximate the derivatives, these require several extra function evaluations, which becomes impractical for problems involving expensive objective functions. In these cases, it becomes necessary to resort to derivative-free algorithms, which, instead of exploiting information about derivatives, treat the objective functions as \textit{black-boxes} and use the result of previously evaluated solutions to guide the search~\cite{Rios2013}.

Derivative-free optimization algorithms, commonly known as black-box optimization algorithms within the architectural community \cite{Wortmann2016BBO},
can treat the results of performance simulations as the functions to optimize and, consequently, avoid the difficulty of deriving analytical formulas describing building performance \cite{Machairas2014}.

For the past decades, the constant development and improvement of derivative-free optimization resulted in the creation of algorithms with different underlying assumptions and different properties. Although there is no standardized classification for these optimization algorithms~\cite{Rios2013, Wortmann2017ADO}, it is possible to group them according to their main mechanisms and ideas. This dissertation follows the classification proposed in the context of architectural design~\cite{Wortmann2015AdvSBO}, which first subdivides the algorithms according to the search strategy's determinism, namely, metaheuristics and iterative methods, and only then proceeds to partition the latter into direct-search and model-based methods, based on the function explored to guide the search. Albeit the apparent chasm between these classifications, some algorithms draw ideas from distinct classes, thus emphasizing not only the blurred lines of such categorizations, but also the difficulties that lie within the definition of more standardized classifications. 

The following subsections describe the classes of algorithms that will be explored in this dissertation, including the three classes of derivative-free algorithms: direct-search, metaheuristics, and model-based. For each section, a summarized explanation about the most relevant algorithms will also be provided. 

\subsection{Direct-search Algorithms}
Although there seems to be no precise definition for direct-search algorithms, these are often identified as algorithms that iteratively: (1) evaluate a sequence of candidate solutions, proposed by a deterministic strategy; and (2) select the best solution obtained up to that time \cite{Conn2009}. They are regarded as valuable tools to address complex optimization problems, not only because most of them were proved to rely on solid mathematical principles, but also due to their good performance at initial stages of the search process~\cite{Rios2013}. 

The main limitations of these algorithms is their performance deterioration with the increase on the number of input variables and their slow asymptotic convergence rates as they get closer to the optimal solution~\cite{Kolda2003}. Despite the existence of algorithms and benchmarks comparing \ac{SOO} direct-search algorithms \cite{Wortmann2017GABESTCHOICE,Waibel2018}, only recently have these started to appear in the context of \ac{MOO}~\cite{Custodio2010,Custodio2018}. 

Undoubtedly, one of the most relevant direct-search algorithms is the \textit{\ac{NMS}} \cite{Nelder1964}. \textit{\ac{NMS}} is a local \ac{SOO} direct-search algorithm that exploits a simplex to guide the search towards a locally optimal solution. The \textit{\ac{NMS}} algorithm envelopes a region of the design space using a simplex, which is a generalization of a triangle to arbitrary dimensions, i.e., a triangle in two dimensions, a tetrahedron in three dimensions, etc. The simplex is then successively modified using operations, like reflection, expansion, contraction, and shrinking, that iteratively replace the simplex's worst vertex values. % http://www.scholarpedia.org/article/Nelder-Mead_algorithm
Unlike other direct-search algorithms, \textit{\ac{NMS}} requires no more than two function evaluations per iteration, except when applying the shrinking operation. The initial simplex highly influences the performance of the algorithm: while smaller initial simplices often lead to local searches that converge towards a local optimum, larger initial simplices allow the algorithm to cover a larger extent of the solution space, thus becoming more robust to local optima.

Another interesting simplex-based algorithm is \textit{SUBPLEX} \cite{Rowan1990}, which attempts to overcome the \textit{\ac{NMS}} difficulties when addressing higher dimensional problems by decomposing the problem in low\nobreakdash-\hspace{0pt}dimensional
subspaces. To that end, \textit{SUBPLEX} subdivides the design space in low-dimensional subspaces and then applies the \textit{\ac{NMS}} algorithm to the most promising subspaces, in order to seek for a better solution. %In contrast to \ac{NMS}, which has difficulties in high-dimensional problems, \textit{SUBPLEX} reduces the limitations through the decomposition of the problem in low-dimensional subspaces which are more efficiently optimized by \ac{NMS}.
In the same vein, \textit{\ac{PRAXIS}} \cite{Brent1973} also decomposes the problem in smaller ones, by considering each dimension of the solution space separately. Concretely, \textit{\ac{PRAXIS}} moves from one solution to another by iteratively finding better solutions in every other dimensions and then combining them into a candidate solution. 

Besides the local algorithms discussed so far, \textit{\ac{DIRECT}} is a very promising global algorithm. \textit{\ac{DIRECT}} \cite{Jones1993DIRECT} is a \ac{SOO} algorithm which recursively subdivides the design space into smaller multidimensional hyper-rectangles, each represented by a solution in their centre. For each solution, the objective function is evaluated, thus yielding an estimate of the quality of each rectangle. Based on these values, \textit{\ac{DIRECT}} focus the search on more promising regions of the design space, further subdividing those. To minimize the overall number of function evaluations, \cite{Gablonsky2001} suggested a modification to \textit{\ac{DIRECT}} to make it more efficient for functions with few local optima and a single global optimum. This modified algorithm, called \textit{\ac{DIRECT}-L}, differs from the original one by grouping the hyper-rectangles based on the size of the longest rectangle side and by allowing at most one subdivision in each group, i.e., at most one hyper-rectangle of each group can be subdivided in each iteration. These modifications to the original algorithm promote the reduction of the number of divisions, which has a direct impact on the overall number of function evaluations.

\textit{\ac{DMS}} \cite{Custodio2010} is a global multi-objective direct-search algorithmic framework which combines the main ideas of directional direct-search algorithms with the Pareto dominance concepts. In simple terms, \textit{\ac{DMS}} maintains a list of feasible nondominated solutions and their associated step sizes. Then, it iteratively selects nondominated solutions from this list and evaluates a few solutions along a predefined set of directions located at a distance determined by the solution's step size. If during the exploration better solutions are found, then the list is updated. On the other hand, if no better solutions are found, the associated step size is reduced. The process then repeats. This algorithmic framework extends to \ac{MOO} the directional type direct-search algorithms, such as pattern search \cite{Kolda2003}, among others. In addition to \textit{\ac{DMS}}, we refer \cite{Custodio2018} to the interested reader for a more recent direct-search \ac{MOOA}.

Overall, direct-search algorithms are not as popular as other classes of derivative-free algorithms. Nevertheless, their convergence proofs and the recent developments in the field of \ac{MOO} make this class very appealing for \ac{BPO}. 

\subsection{Metaheuristics Algorithms}

Metaheuristics are algorithms that employ simple mechanisms, called heuristics, to locate good solutions in complex design spaces, while considering the trade-off among precision, quality, and computational effort of the solutions \cite{Glover2003Metaheuristics}. These algorithms often rely on randomization, and biological or physical analogies to perform robust searches and to escape local optima. Additionally, through these heuristics, the designer is able to increase the overall performance by adding domain-specific knowledge. Moreover, their non-deterministic and inexact nature confer them the ability to effortlessly handle complex and irregular objective functions \cite{Wortmann2017GABESTCHOICE}.

These algorithms often rely on randomization, and biological or physical analogies, to perform robust searches and escape local optima~\cite{Glover2003Metaheuristics}. Additionally, their stochastic nature confers them the ability to effortlessly handle complex and irregular objective functions, to adapt to \ac{MOO} contexts, or to provide domain-specific knowledge through heuristics \cite{Wortmann2017GABESTCHOICE}.

\subsection{Model-based Algorithms}
	\section{Algorithmic Optimization in Architecture}
\label{sec: Methodology}

With the ability to quickly update a design, to generate the corresponding analytical model, and, finally, to automatically evaluate the design in an analytical tool, it becomes possible to implement automated optimization processes. We name these processes \ac{AO}.

\ac{AO} treats the analyses' results for different variations of the design as the functions to optimize, i.e., as the objective functions. These functions have a domain which corresponds to the range of acceptable designs as specified by the architect. Moreover, since we do not know their mathematical form, these objective functions are often treated as black- boxes, and, as a result, information about the their derivatives cannot be extracted. For this reason, methods depending on function derivatives cannot be used to address this problem. Instead, we use derivative-free (or black-box) algorithms, which can be divided into three distinct subclasses: direct-search, metaheuristics, and model-based \cite{Conn2009,Glover2003Metaheuristics,Koziel2011}.


In \cref{ssec:ad,ssec:aa}, we discussed how the architectural design paradigm has incrementally grown to develop the mechanisms to quickly (1) update a design, (2) generate the corresponding analytical model, and (3) automatically evaluate the design in an analytical tool.% and collect its results. 

These mechanisms laid down the foundations for automated optimization processes. By extending the Algorithmic Design (\ac{AD}) and Algorithmic Analysis (\ac{AA}) approaches to include optimization mechanisms, we are able to apply automatic optimization processes that aim to improve (or even optimize) designs' performance. 


\Cref{fig:algorithmicoptimization} illustrates a possible approach for introducing automated optimization processes in the architectural worfklow. In this approach, we introduce an optimizer component that is responsible for searching the design space and generating new values for the design's parameters. These values are then communicated to the \ac{AD} tool, which generates the analytical models and evaluates them in the corresponding analytical tools. After being evaluated, the analytical tools communicate the evaluations' results to the \ac{AD} tool, which, in turn, forwards them to the optimizer. Based on these results, the optimizer generates another set of values for the design's parameters and this process is repeated until a stopping criterion (e.g., evaluations or time limit, solution's quality) is met. The communications between each component are encoded within the \ac{AD} tool, therefore incurring no additional efforts for the architect.

\begin{figure}[htbp]
	\centering
	\includegraphics[width=\columnwidth]{../report/Images/Solution/algorithmic_optimization.png}
	\caption[Algorithmic Optimization workflow]{Algorithmic Optimization workflow. In this workflow, the architect only interacts with an \ac{AD} tool to create the initial design, to specify the analysis tools, and to specify the optimization parameters.}
	\label{fig:algorithmicoptimization}
\end{figure}

In order to benefit from the \ac{AO} approach, we combine the optimization framework developed in this dissertation with an \ac{AD} tool to provide an alternative to easily address a wide variety of \ac{BPO} problems. To this end, architects are required to: (1) create the \ac{AD} model reflecting their design's intents; (2) select the performance aspects to optimize and, thus, the analysis tools to be used (e.g., lighting, thermal, structural, costs), and, finally, (3) to select, if necessary, the parameters of the optimization process (e.g., algorithm, algorithm's parameters).


Even though the provided example presents a textual-based \ac{AO} approach, the developed optimization framework can easily be integrated within a visual \ac{AD} tool, like Grasshopper. 

One important aspect that results from this combination is the fact that architects are able to use different optimization algorithms and, consequently, to select an optimization algorithm that better suits their problems. To bridge the gap between the lack of knowledge or experience and the suitability of optimization algorithms for each problem, the optimization framework also provides automated testing mechanisms. These mechanisms enable the sequential execution of multiple optimization algorithms for a specified amount of evaluations, as well as each algorithm's performance measures. This feature is particularly important in the architectural context~\cite{Wortmann2016BBO,Hamdy2016}, as it promotes more informed decisions towards the selection of more appropriate optimization algorithms.


Due to the visual nature of architects and the generalized lack of confidence in optimization processes, the visualization of evaluated design solutions is of great importance, as it allows architects to explore and corroborate the optimization results. To this end, the optimization framework provides post-processing and visual mechanisms (see \cref{fig:postprocessing}) that, when combined with an \ac{AD} tool, allow the architect to click on the evaluated solutions and instantly visualize the corresponding design in a 3D modeling tool. 

These visual mechanisms also promote a better comprehension of the optimization process, as they allow architects to explore and visualize the results of the optimization, in real-time, to get a clearer perspective. Besides reasoning and creating logical patterns that allow them to explain the obtained results, architects can also learn more about their designs' behavior regarding different performance aspects and, potentially, about the optimization algorithm itself.

Finally, visualization and processing mechanisms can also be useful not only to detect errors or incoherences (e.g., in the optimization model) early in the optimization process, but also to reduce the overall optimization time, i.e.,  provide the architect with enough information to stop the optimization process sooner. For some problems, obtaining an optimum is not strictly necessary. Instead, a close to optimal or good solution suffices. Having a framework which interactively updates the information about the optimization process is particularly useful for those problems, since the user can explore and visualize the already evaluated candidate solutions and decide whether one of them suffices, even if it is not an optimum. 

	\section{Case Studies Evaluation}
\label{sec:resul}


	\section{Conclusions}
\label{sec:concl}

Nowadays, with the threat of climate change, resource depletion, and worldwide urbanization, it is not enough to construct well-designed buildings, it is also necessary to optimize them \cite{Wortmann2015AdvSBO}. Architectural practices have, therefore, grown to incorporate considerations about the building's performance in various aspects. The development of computational simulation tools empowered designers with the ability to simulate and estimate a building’s performance. The emergence of these tools and the raising concerns about the environmental and economic impact of buildings led to the development of new design approaches, such as \ac{PBD}, which seek more efficient design solutions by considering the designs’ performance. Taking \ac{PBD} a step further, optimization has unveiled a new performance-based approach called \ac{BPO}. 

Unfortunately, traditional \ac{BPO} methodologies require the evaluation of different design variations, which, in turn, implies spending a large amount of time with the manual application of changes to the design and often leading to difficulties when modeling complex geometry. Moreover, in order to evaluate a design's performance, the corresponding analytical models must be produced, which also comprises a time-consuming and tiresome task. The emergence of algorithmic-based paradigms, like \ac{AD} and \ac{AA}, enabled the implementation of automated optimization processes, as they allow architects to generate multiple design variants with little effort, to automatically produce the corresponding analytical models and to automatically evaluate their performance. 

Optimization algorithms can be coupled with the previously mentioned algorithmic approaches and simulation tools to more efficiently seek for optimal design solutions. Given that a single evaluation may take a considerable amount of time to complete, in order to speed up the optimization process, it becomes necessary to identify the optimization algorithms capable of handling the computationally complex problems that characterize \ac{BPO}, and to devise strategies for its efficient application in architecture. Often disregarded, the selection of the appropriate optimization algorithm can have a significant impact in the overall efficiency of optimization processes and on the quality of the results \cite{Wolpert1997NFLT}.

However, most \ac{BPO} practitioners adopt the simplest available algorithm, such as \acp{EA} \cite{Evins2013}, which typically require hundreds or thousands of evaluations - an infeasible scenario for most \ac{BPO} problems. Conversely, direct-search and model-based algorithms are more promising, frequently yielding better results in fewer evaluations \cite{Waibel2018}. Particularly, model-based algorithms can considerably reduce the time spent in optimization \cite{Wortmann2017GABESTCHOICE}, and, in fact, results show that these algorithms exhibit an average reduction of $50\%$ on the total time spent in optimization.

Along these lines, we addressed optimization algorithms specifically tailored for handling simulation-based optimization problems, where the computational effort of simulation often restricts the number of function evaluations to a few dozens or hundreds. We introduced \ac{AO}, an extension to the \ac{AD} and \ac{AA} approaches, that combines an optimization framework with an \ac{AD} tool to address various design optimization problems, including \ac{BPO}. Results show that no single class, subclass, or algorithm excels at every problem, thus corroborating the \acp{NFLT} for optimization \cite{Wolpert1997NFLT}. 

Moreover, even though other tools rely on \ac{GA}-based algorithms, these rarely achieved the best performance in the evaluated case studies. Instead, other categories, such as model-based or direct-search algorithms, yielded better results. Different factors could change the obtained results (e.g., a different configuration of the algorithms or a lucky random step). Therefore, and contrarily to current architectural practices, we conclude that distinct algorithms behave differently according to the problems' characteristics and that architects should first test various algorithms for a small number of evaluations or for a short amount of time. 
Furthermore, we also conclude that while global optimization algorithms are quicker to converge towards optimal solutions when no additional information is known, local algorithms can be quicker if provided with good starting points and, therefore, should be considered when such information is available.

Regarding the evaluation of different \acp{MOO} algorithms, the lack of consensus regarding the appropriate way to measure their quality makes it difficult to quantify the suitability of each algorithm for \ac{MOO} problems. We conclude that algorithms' quality should be measured through the combination of Pareto front plots and multiple \ac{MOO} performance indicators. Particularly, some indicators should provide a measure of the diversity of the nondominated solutions across the solution space, while others should measure the overall accuracy of the results.

To overcome the identified limitations, the proposed optimization framework includes different categories of optimization algorithms and facilitates their application by abstracting them under a common interface, thus promoting automated optimization processes. To further facilitate the selection of the most appropriate algorithm, it also includes mechanisms to effortlessly test multiple algorithms. The framework was evaluated in the context of two \ac{BPO} problems, which demonstrated its ability to solve real architectural problems characterized by computationally complex objective functions.

%Overall, optimization can be very beneficial for architecture. The combination of simulation tools, algorithmic-based approaches, and optimization algorithms enables the automation of optimization processes within architectural practices. Despite its benefits, the incorrect application of these optimization processes can lead to poor results. Moreover, distinct architectural design problems benefit from different optimization algorithms. However, most architectural optimization tools focus on the same subset of algorithms, which are rarely the most adequate. The proposed framework circumvents these limitations by presenting several optimization algorithms with different characteristics, thus fostering better optimization practices. Finally, the proposed \ac{AO} methodology was shown to benefit the architectural practice, as demonstrated by the case studies.

% REFERENCES

% Produces the bibliography section when processed by BibTeX
%
% Bibliography style
% > entries ordered alphabetically
%\bibliographystyle{plain}
% > unsorted with entries appearing in the order in which the citations appear.
%\bibliographystyle{unsrt}
% > entries ordered alphabetically, with first names and names of journals and months abbreviated
\bibliographystyle{abbrv}
% > entries ordered alphabetically, with reference markers based on authors' initials and publication year
%\bibliographystyle{alpha}

% External bibliography database file in the BibTeX format (ExtendedAbstract_ref_db.bib)
\bibliography{Thesis-MSc-Bibliography}

\end{document}


