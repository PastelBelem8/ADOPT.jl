% #############################################################################
% Abstract Text
% !TEX root = ../main.tex
% #############################################################################
% use \noindent in firts paragraph
\noindent 
%  O resumo analítico, também designado por resumo ou abstract, descreve o objectivo, o conteúdo do trabalho e as conclusões. Deve ser escrito em português e inglês, com um máximo de 250 palavras cada e acompanhado de 4 a 6 palavras-chave;

The building sector currently presents one of the largest economic and environmental footprints. Optimization can minimize this impact by finding more efficient building variants, prior to their construction. Building design optimization spurs the exploration of (1) algorithmic approaches, to generate multiple design variants, (2) simulation tools, to evaluate design's performance regarding distinct aspects (e.g., acoustics, thermal, structural, costs), and (3) optimization algorithms, to seek more efficient design variants. Unfortunately, despite the existence of several optimization algorithms, their low availability within architectural tools and the lack of experience/knowledge often drive architects towards the application of the simplest available algorithm. However, this algorithm is rarely the most efficient option for addressing a specific problem, often struggling to find more efficient designs or comprising a time-consuming process. The impacts of a poor algorithm selection are even more exasperate when considering the optimization of multiple conflicting design aspects simultaneously. This dissertation studies optimization approaches and algorithms specifically tailored for addressing simulation-based optimization problems. In particular, we develop an optimization framework that incorporates different optimization approaches and algorithms. We test the different optimization approaches and algorithms presented in the framework in the context of three architectural case studies. Obtained results reveal the performance impacts associated with a pondered choice of an algorithm for a specific problem. Moreover, results corroborate the idea that different optimization approaches and algorithms should be tested in order to infer the one that better fits the problem to address. 
