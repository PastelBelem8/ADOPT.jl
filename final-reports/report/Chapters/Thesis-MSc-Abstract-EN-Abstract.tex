% #############################################################################
% Abstract Text
% !TEX root = ../main.tex
% #############################################################################
% use \noindent in firts paragraph
\noindent 
%  O resumo analítico, também designado por resumo ou abstract, descreve o objectivo, o conteúdo do trabalho e as conclusões. Deve ser escrito em português e inglês, com um máximo de 250 palavras cada e acompanhado de 4 a 6 palavras-chave;

The building sector presents one of the largest economic and environmental footprints. Building performance optimization can minimize this impact by combining (1) algorithmic approaches, to generate multiple building design variants, (2) simulation tools, to evaluate building's performance regarding distinct aspects, and (3) optimization algorithms, to seek more efficient building designs. Unfortunately, despite the existence of several optimization algorithms, their application to architectural optimization problems is not well-studied and this often drives architects towards the application of the simplest available algorithm. However, this rarely is the most efficient option for addressing a specific problem, in particular, due to the time-intensive simulations required to evaluate building designs. As a result, poor algorithm selection might lead to unacceptable optimization times and less efficient designs. 

This dissertation addresses optimization algorithms specifically tailored for handling simulation-based optimization problems. In particular, we develop and assess an optimization framework in the context of three architectural case studies involving single- and multi-objective optimization of simulation-based lighting, structural, and cost aspects of buildings. Obtained results reveal that solutions' quality and the time spent in optimization are strongly dependent on the algorithm's choice. Thus, different algorithms should be tested in order to determine the one that better fits an optimization problem. Finally, this dissertation shows the benefits of optimization to reduce the impact of the building sector and motivates its introduction as an indispensable phase of the architectural design process.
