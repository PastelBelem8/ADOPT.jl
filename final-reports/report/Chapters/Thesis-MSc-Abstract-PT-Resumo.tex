% #############################################################################
% RESUMO em Português
% !TEX root = ../main.tex
% #############################################################################
% use \noindent in firts paragraph
\noindent 
O sector da construção representa uma grande pegada económica e ambiental. A optimização baseada no desempenho de edifícios pode minimizar este impacto ao combinar (1) abordagens algorítmicas, para geração de múltiplas variações de design de um edifício, (2) ferramentas de simulação, para avaliar o desempenho de um design em relação a vários aspectos, e (3) algoritmos de optimização para procurar os designs mais eficientes. Apesar da existência de vários algoritmos de optimização, a sua aplicação a problemas em arquitectura não foi muito explorada, levando os arquitectos a optarem por algoritmos facilmente acessíveis e simples. No entanto, estes raramente são as melhores escolhas para resolver problemas específicos, especialmente, porque para avaliar os designs é necessário recorrer a simulações computacionalmente pesadas. Como consequência, a má escolha destes algoritmos leva a tempos de execução inaceitáveis e a designs menos eficientes.

Esta dissertação foca-se em algoritmos especialmente projectados para problemas de optimização baseados em simulação. Em particular, nós desenvolvemos e avaliamos uma framework de optimização no contexto de três casos de estudo em arquitectura que envolvem a optimização único e múltiplo objectivo de vários aspetos dos edifícios. Os resultados revelam o impacto da escolha do algoritmo na qualidade das soluções e no tempo total de execução. Assim, diferentes algoritmos devem ser testados de modo a determinar aquele que melhor satisfaz um problema de optimização. Finalmente, esta dissertação demonstra os benefícios da optimização para reduzir o impacto do sector de construção e motiva a sua introdução como uma fase indispensável no processo de design arquitectónico.

