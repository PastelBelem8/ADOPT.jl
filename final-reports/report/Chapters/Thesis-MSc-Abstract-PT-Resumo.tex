% #############################################################################
% RESUMO em Português
% !TEX root = ../main.tex
% #############################################################################
% use \noindent in firts paragraph
\noindent 
O sector da construção representa uma das maiores pegadas económicas e ambientais. Optimização baseada no desempenho de edifícios pode minimizar este impacto ao combinar (1) abordagens algorítmicas, para geração de múltiplas variações do design do edifício, (2) ferramentas de simulação, para avaliar desempenho de um design em relação a vários aspectos, e (3) algoritmos de optimização para procurar os designs mais eficientes. Infelizmente, apesar da existência de vários algoritmos de optimização, a sua aplicação a problemas em arquitectura ainda não foi muito explorada, levando os arquitectos a frequentemente optarem por algoritmos facilmente acessíveis e simples. No entanto, esta raramente é a escolha mais eficiente para resolver um problema específico, em particular, porque para avaliar o design de edíficios é necessário recorrer a simulações computacionalmente pesadas. Como consequência, a má escolha de um algoritmo de optimização pode levar a tempos de execução inaceitáveis, assim como a designs menos eficientes.

Esta dissertação foca-se em algoritmos de optimização especialmente desenhados para lidar com problemas de otimização baseados em simulação. Em particular, nós desenvolvemos e avaliamos uma framework de optimização no contexto de três casos de estudo em arquitectura que involvem a optimização único e múltiplo objectivo dos aspetos lumínicos, estruturais, e de custo dos edíficios. Os resultados obtidos revelam que a qualidade das soluções e o tempo dispendido na optimização dependem fortemente da escolha de algoritmo. Deste modo, diferentes algoritmos devem ser testados de modo a determinar aquele que melhor satisfaz um problema de optimização. Finalmente, esta dissertação demonstra os benefícios da optimização para reduzir o impacto do sector de construção e motiva a sua introdução como uma fase indispensável no processo de design arquitectónico.

