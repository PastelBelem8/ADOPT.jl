% #############################################################################
% RESUMO em Português
% !TEX root = ../main.tex
% #############################################################################
% use \noindent in firts paragraph
\noindent 
O sector da construção representa actualmente uma das maiores pegadas económicas e ambientais. A optimização pode minimizar este impacto ao permitir encontrar variações mais eficientes de edíficios antes da sua construção. A optimização do projecto de edíficios estimula a exploração de (1) abordagens algorítmicas para gerar múltiplas variações de um projecto, (2) ferramentas de optimização para avaliar o desempenho de um projecto com respeito a diferentes aspectos (p. ex., acústico, térmico, estrutural, custo) e (3) algoritmos de optimização para procurar variações mais eficientes de um projecto. Infelizmente, apesar da existência de vários algoritmos de optimização, existem poucos estudos sobre a aplicação destes algoritmos a problemas de optimização arquitectónica que, acrescido da falta de experiência e/ou conhecimento por partes dos arquitetos, leva os mesmos a aplicarem o algoritmo mais simples. Porém, este algoritmo raramente representa a melhor opção para um problema específico, tipicamente apresentando dificuldades na procura de soluções mais eficientes e/ou constituindo processos computacionalmente complexos. Os efeitos da escolha de um mau algoritmo são ampliados quando se considera a optimização de vários aspectos de projecto, conflituosos entre si, simultâneamente.

Esta dissertação estuda algoritmos de optimização focados na resolução de problemas de otimização baseados em simulações. Em particular, nós desenvolvemos uma ferramenta de otimização que incorpora diferentes abordagens e algoritmos de otimização. A ferramenta foi avaliada no contexto de três casos de estudo arquitectónicos. Os resultados obtidos evidenciam o impacto no desempenho que uma escolha ponderada de um algoritmo pode ter. Adicionalmente, os resultados comprovam a ideia de que diferentes algoritmos de otimização deverão ser testados para determinar qual o mais apropriado para o problema a resolver.