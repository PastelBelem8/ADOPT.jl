% #############################################################################
% This is Appendix A
% !TEX root = ../main.tex
% #############################################################################
\chapter{Algorithms Definitions}
\label{chapter:appendixA}

\todo{REFS e melhorar descrições?}

\section{Sampling Algorithms}
\begin{itemize}
\item \textbf{Monte Carlo Sampling} or Random Sampling
\item \textbf{Stratified Monte Carlo Sampling}
\item \textbf{Latin Hypercube Sampling}
\item \textbf{K-factorial}
\item \textbf{Full-factorial}
\end{itemize}


\section{Direct-Search Algorithms}
\begin{itemize}
	\item \textbf{Nelder-Mead Simplex} (NMS)~\cite{Nelder1964} constructs a geometric figure, known as simplex, to envelope a region of the design space. After creating it, the algorithm moves the geometric figure across the design space by iteratively changing its vertices.
	
	\item \textbf{Subspace simplex} (Subplex)~\cite{Rowan1990} algorithm is an unconstrained local optimization algorithm~\cite{Rowan1990}. As a generalization of the NMS algorithm, SUBPLEX subdivides the design space in low-dimensional subspaces and then applies the NMS algorithm to a set of these subspaces, in order to seek for a better solution. In contrast to NMS, which has difficulties in high-dimensional problems, SUBPLEX reduces the limitations through the decomposition of the problem in low-dimensional subspaces which are more efficiently optimized by NMS.
	
	\item \textbf{Principal Axis} (PRAXIS) moves from one region to other by iteratively applying line search algorithms to each direction. The goal of these line search algorithms is to determine a better solution along a certain dimension.
	
	\item \textbf{DIRECT}~\cite{Jones1993DIRECT} is an algorithm that recursively subdivides the design space into smaller multi-dimensional hyper-rectangles, estimating the quality value of each rectangle. DIRECT uses these values to focus the search on more promising regions of the design space and to further subdivide those in smaller hyper-rectangles. 
	
	\item \textbf{DIRECT-L} is a a local variant of DIRECT claimed to be more efficient for functions with few local optima \todo{(Glabonsky and Kelley 2001)}.
	
	
	\item \textbf{Direct MultiSearch} (DMS) %https://www.mat.uc.pt/~lnv/papers/dms.pdf
	% http://ferrari.dmat.fct.unl.pt/personal/alcustodio/multiglodspaper.pdf


\end{itemize}	

\section{Metaheuristics Algorithms}

A very summarized explanation will be provided for the metaheuristic algorithms studied during this dissertation. Therefore, we suggest \cite{BlumRoli2003Metaheuristics,Glover2003Metaheuristics} for more complete explanations and further literature on these and other metaheuristic algorithms.

\begin{itemize}
	\item \textbf{Controlled Random Search 2} (CRS2) algorithm is a variant of the CRS algorithm for global optimization~\cite{Price1983}. A CRS algorithm is a population-based random search algorithm that creates an initial set of points, the population, which are then randomly evolved by means of heuristic rules. In the original CRS algorithms, heuristic rules modify a point at a time, replacing the worst point with a better one (called trial point), using a technique resemblant of the NMS algorithm \cite{Nelder1964}. Similarly to NMS, CRS algorithms use a simplex, i.e., a generalized triangle in $N$ dimensions, to envelope a region described by a random subset of points in the population. The worst point of the population is replaced with the reflection of the worst point in the simplex~\cite{Kaelo2006CRS2}. The CRS2 variant differs from the original in that it assumes that the worst population point will always be a part of the simplex. The actual version that is made available by Goat is a modified version of the CRS2 algorithm that introduces a local mutation component in an attempt to overcome situations where heuristic rules constantly fail to find a trial points that actually improve the worst point. Fundamentally, this local mutation generates a second trial point that results from the exploitation of the region around the best point in the population~\cite{Kaelo2006CRS2}.  
	
	\item \textbf{Evolutionary Strategy with Cauchy Distribution} (ESCH) \todo{ref (Santos 2010)} creates an initial population that is then iteratively recombined according to a specific distribution function, instead of the uniform distribution, and then mutated. The mutations are the result of combining two different mutation operators - the gaussian variation and the duplication gene mutations.
	
	\item \textbf{Improved Stochastic Racking Evolutionary Strategy} (ISRES) \todo{ref (Runarsson and Yao 2005)} is a global evolution strategy method, i.e., a method that maximizes the suitability of the candidate designs given by a fitness ranking function. The candidate designs are evolved iteratively by combining the application of a mutation rule and differential variation.
	
	\item \textbf{Simulated Annealing} algorithm is a global optimization physical algorithm. Resemblant of hill climbing algorithms, where new candidate solutions are randomly sampled, this algorithm iteratively re-samples the solution space aiming at finding an optimal solution. During the search, the algorithm is propitious to accept the re-sampled solutions with lower performance, according to a probabilistic function that becomes more discerning of the quality of the samples over the execution of the algorithm, thus resembling the natural annealing process~\cite{Brownlee2011}. 
	
	\item \textbf{(Covariance Matrix Adaptation - Evolutionary Strategy)} (CMA-ES) 
\end{itemize}

\todo{Change this}
Regarding \ac{MOO} algorithms. 
For a more thorough description of different \acp{MOEA} and their mechanisms, we suggest \cite{Zhou2011}. 

\begin{itemize}
	\item \textbf{SPEA2}
	
	\item \textbf{NSGA-II}
	
	\item \textbf{Epsilon \ac{MOEA}} ($\epsilon$-MOEA) 
	
	\item \textbf{Hypervolume Estimation Algorithm for \ac{MOO}} (HypE) 
	Unfortunately, algorithms incorporating the Pareto dominance relation and diversity measures appear to have difficulties in optimization scenarios with more than two objectives, which spurred the development of algorithms using other quality measures, including quality indicators. To overcome the objective limitation and provide the user with the flexibility to use a more efficient algorithm, Octopus provides the option to use HypE, an algorithm that explores the \ac{HV} indicator to rank individuals. Particularly, to minimize time penalties associated with the computation of the \ac{HV}, in scenarios with more than three objectives, HypE uses Monte Carlo simulations to estimate the \ac{HV} value of each individual~\cite{Zitzler2011HypE}. 
	
	\item \textbf{\ac{MOEA} with Decomposition} (MOEA/D)
	
	\item \textbf{Pareto Archived Evolution Strategy} (PAES)
	
	\item \textbf{Pareto Envelope-Based Selection Algorithm 2} (PESA2)
	
	\item \textbf{Speed-constrained Multi-objective \ac{PSO}} (SMPSO)
	
	\item \textbf{OMOPSO}
	
\end{itemize}


\section{Model-based Algorithms}
\begin{itemize}
	\item 	Since its invention, multiple implementations of \ac{RBF} have been proposed. RBFOpt implements two well-known versions, namely, the Gutmann's~\cite{Gutmann2001} and the Regis and Shoemaker's~\cite{Regis2007}, commonly known as MSRSM. These techniques differ in the search strategy for the next candidate solution to be evaluated using the original objective function. The former uses the solution, which is likely to yield the largest improvement in the surrogate's accuracy, whereas the latter tries to balance the surrogate's accuracy amelioration with the exploitation of promising solutions, using other search strategies, such as genetic algorithms, sampling algorithms, or other mathematical solvers~\cite{Wortmann2017Opossum}. Moreover, RBFOpt provides five different types of radial basis function: linear, multi-quadratic, cubic, thin plate spline, and automatic selection, which are also provided in the Opossum interface.
\end{itemize}