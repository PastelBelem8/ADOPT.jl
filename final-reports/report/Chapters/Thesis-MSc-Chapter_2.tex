% #############################################################################
% This is Chapter 2
% !TEX root = ../main.tex
% #############################################################################
% Change the Name of the Chapter i the following line
\fancychapter{Background}
\cleardoublepage
% The following line allows to ref this chapter
\label{chap:back}

	The development of an algorithmic-based framework for optimization, applicable to architectural domains, requires a careful review over the current literature on \ac{BPO} practices and limitations. 
	
	Firstly, the \textit{ad-hoc} nature of the functions used for performance assessment in \ac{BPO} motivates the application of a special class of optimization algorithms, the derivative-free algorithms. Within this class, different categories emerge, emphasizing algorithms' different properties and search strategies. Each algorithm is able to potentially increase the effectiveness of an optimization process, depending on the characteristics of the considered problem. 
	
	Secondly, there are multiple approaches to optimization that might be considered. Generally, \ac{BPO} practices include the simultaneous optimization of multiple aspects. However, they often opt for simpler  specifications, often disregarding all but one of the initial considered aspects. 

	Finally, currently available architectural design optimization tools explore the parametric models produced in visual programming environments, such as Grasshopper and Dynamo. These graphical environments are implemented as plug-ins, which are tightly integrated with a \ac{CAD} and a \ac{BIM} tools, respectively. As a result, the connection between optimization tools and visual design workflows becomes seamless and friendlier. These optimization tools usually expose a \textit{ready-to-run} interface, which is very appealling to most \ac{BPO} practitioners~\cite{Cichocka2017SURVEY}.
	
% #############################################################################
\section{Derivative-Free Optimization}


% #############################################################################
\section{Single-Objective Optimization}

% #############################################################################

%Sed pulvinar, \enquote{felis id consectetuer} malesuada, enim nisl mattis elit, a facilisis tortor nibh quis leo \Cref{tab:streamingtech}.
%
%\begin{table}[htb]
%\centering
%\normalsize
%{\footnotesize
%    \caption{Streaming Technologies Comparison}
%    \label{tab:streamingtech}
%    \begin{tabular}{ | c | c | c | c |}
%    \hline
%    & Dynamic & Smooth & HLS\\
%    & Streaming & Streaming & \\ \hline \hline
%
%    Streaming Protocol & RTMP & HTTP & HTTP \\
%    %\textbf{Protocol} & & & \\ 
%    \hline
%    
%    Video Codec & H.264, VP6 & H.264 & H.264 \\ 
%    %\textbf{Codec} & &  & \\ 
%    \hline
%    
%    Audio Codec & AAC, MP3 & WMA, AAC & AAC, MP3  \\
%    %\textbf{Codec} & & & \\ 
%    \hline
%    
%    Container Format & MP4, FLV, & MP4 & MPEG2-TS \\
%    %\textbf{Format} & F4V & & \\ 
%    \hline
%    
%     iOS & NO & YES & YES \\ \hline
%     
%    Android & NO & YES & YES \\ \hline
%    
%    \end{tabular}
%    }
%\end{table} 
%
%Suspendisse vestibulum dignissim quam. Integer vel augue. Phasellus nulla purus, interdum ac, venenatis non, varius rutrum, leo. Pellentesque habitant morbi tristique senectus et netus et malesuada fames ac turpis egestas \cite{RFC-VP8}. Duis a eros. Class aptent taciti sociosqu ad litora torquent per conubia nostra, per inceptos hymenaeos. Fusce magna mi, porttitor quis, convallis eget, sodales ac, urna \cite{Chiang:2011fk}. \textcolor{violet}{\Cref{tab:spreadtb} illustrates the use of a Spreadsheet-like table producing calculations by columns and by lines (observe the code).} 
%
%\begin{table}[htb]
%\centering
%    \caption{A nice Spreadsheet using package ``spreadtab''. Notice the calculations.}
%    \label{tab:spreadtb}
%\begin{spreadtab}{{tabular}{rr|r}} 
%22       & 54       & a1+b1 \\
%43       & 65       & a2+b2 \\ 
%49       & 37       & a3+b3 \\
%\hline
%a1+a2+a3 & b1+b2+b3 & a4+b4
%\end{spreadtab}
%\end{table} 
% #############################################################################
\subsection{Optimization Tools in Architecture}

% #############################################################################
\subsubsection{Galapagos}

% #############################################################################
\subsubsection{Goat}
% #############################################################################
\section{Multi-Objective Optimization}
Nunc tincidunt convallis tortor. Duis eros mi, dictum vel, fringilla sit amet, fermentum id, sem. Phasellus nunc enim, faucibus ut, laoreet in, consequat id, metus. Vivamus dignissim \cite{Moscoso:2011fk}. \textcolor{violet}{\Cref{tab:comp_arch} is automatically compressed to fit text width. You can use \url{https://www.tablesgenerator.com} to produce these tables, and then copy the \LaTeX\space code generated to paste in the document.}

% #############################################################################
\subsection{Experimental Approach}

% #############################################################################
\subsection{Priori Articulation Approach}

% #############################################################################
\subsection{Pareto-Based Approach}

% #############################################################################
\subsection{Metrics for Multi-Objective Optimization}

% #############################################################################
\subsection{Optimization Tools in Architecture}

\subsubsection{Octopus}
\subsubsection{Opossum}
\subsubsection{Optimo}

%\begin{table}[h]
%\centering
%\caption{Comparison between today's and target Architectures of Telcos}
%\label{tab:comp_arch}
%\resizebox{\textwidth}{!}{%
%\begin{tabular}{|
%>{\columncolor[HTML]{ECF4FF}}l |l|
%>{\columncolor[HTML]{E2FFC9}}l |l|}
%\hline
%\multicolumn{2}{|c|}{\cellcolor[HTML]{ECF4FF}Today}                                                                                                             & \multicolumn{2}{c|}{\cellcolor[HTML]{E2FFC9}Target}                                                                                  \\ \hline
%Rigid     & \begin{tabular}[c]{@{}l@{}}Each evolutionary requirement involves \\ development of multiple components, \\ interfaces, platforms,etc.\end{tabular} & Flexible       & \begin{tabular}[c]{@{}l@{}}It is possible to modify or add new\\ functionalities rapidly.\end{tabular}              \\ \hline
%Slow      & \begin{tabular}[c]{@{}l@{}}Development of a new application takes \\ months or years.\end{tabular}                                                  & Fast           & \begin{tabular}[c]{@{}l@{}}Development of a new application takes \\ weeks instead of months or years.\end{tabular} \\ \hline
%Closed    & \begin{tabular}[c]{@{}l@{}}Limited integration with external\\ environments.\end{tabular}                                                           & Open           & \begin{tabular}[c]{@{}l@{}}It is simple to integrate internal,\\ applications with external entities.\end{tabular}  \\ \hline
%Complex   & \begin{tabular}[c]{@{}l@{}}Heterogeneous technologies, obsolescence, \\ lack,of standards, high redundancy.\end{tabular}                            & Standardised   & Use of homogeneous architectural models.                                                                            \\ \hline
%Expensive & \begin{tabular}[c]{@{}l@{}}High Capex (for new service development) \\ and,high,Opex (to ensure running of IT).\end{tabular}                        & Cost-Effective & Capex and Opex are optimised.                                                                                       \\ \hline
%\end{tabular}
%}
%\end{table}

Cras lobortis tempor velit. Phasellus nec diam ac nisl lacinia tristique. Nullam nec metus id mi dictum dignissim. Nullam quis wisi non sem lobortis condimentum. Phasellus pulvinar, nulla non aliquam eleifend, tortor wisi scelerisque felis, in sollicitudin arcu ante lacinia leo.