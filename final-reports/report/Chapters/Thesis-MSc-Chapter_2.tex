% #############################################################################
% This is Chapter 2
% !TEX root = ../main.tex
% #############################################################################
% Change the Name of the Chapter i the following line
\fancychapter{Background}
\cleardoublepage
% The following line allows to ref this chapter
\label{chap:back}




One important classification is regarding the cardinality of the solutions sought by optimization processes, thus yielding the continuous and discrete optimization categories. In the former, the optimal solutions lie in a potentially infinite set of candidate solutions, whereas in the latter, the optimal solutions lie in a finite set. Optimization problems can also be classified as constrained or unconstrained, depending on whether the models explicitly define constraints or not.

Optimization can also be distinguished in terms of the aim of the search that is performed, particularly, whether it is global or local. In local optimization the search process strives to find a solution that is locally optimal, i.e., for which its value is better than all other points in its vicinity. The points that satisfy the previous property are known as local optima. On the other hand, there are optimization processes that strive to find the globally optimal solutions, i.e., the best of all the local optima. 

% #############################################################################
\section{Single-Objective Optimization}
Cras dictum. Maecenas ut turpis. In vitae erat ac orci dignissim eleifend. Nunc quis justo. Sed vel ipsum in purus tincidunt pharetra \cite{MacAulay:2005fk}. Sed pulvinar, felis id consectetuer malesuada, enim nisl mattis elit, a facilisis tortor nibh quis leo. Sed augue lacus, pretium vitae, molestie eget, rhoncus quis, elit \cite{Schwarz:2007lr}. Donec in augue. Fusce orci wisi, ornare id, mollis vel, lacinia vel, massa. Pellentesque habitant morbi tristique senectus et netus et malesuada fames ac turpis egestas..

% #############################################################################
\subsection{Derivative-Free Optimization}

%Sed pulvinar, \enquote{felis id consectetuer} malesuada, enim nisl mattis elit, a facilisis tortor nibh quis leo \Cref{tab:streamingtech}.
%
%\begin{table}[htb]
%\centering
%\normalsize
%{\footnotesize
%    \caption{Streaming Technologies Comparison}
%    \label{tab:streamingtech}
%    \begin{tabular}{ | c | c | c | c |}
%    \hline
%    & Dynamic & Smooth & HLS\\
%    & Streaming & Streaming & \\ \hline \hline
%
%    Streaming Protocol & RTMP & HTTP & HTTP \\
%    %\textbf{Protocol} & & & \\ 
%    \hline
%    
%    Video Codec & H.264, VP6 & H.264 & H.264 \\ 
%    %\textbf{Codec} & &  & \\ 
%    \hline
%    
%    Audio Codec & AAC, MP3 & WMA, AAC & AAC, MP3  \\
%    %\textbf{Codec} & & & \\ 
%    \hline
%    
%    Container Format & MP4, FLV, & MP4 & MPEG2-TS \\
%    %\textbf{Format} & F4V & & \\ 
%    \hline
%    
%     iOS & NO & YES & YES \\ \hline
%     
%    Android & NO & YES & YES \\ \hline
%    
%    \end{tabular}
%    }
%\end{table} 
%
%Suspendisse vestibulum dignissim quam. Integer vel augue. Phasellus nulla purus, interdum ac, venenatis non, varius rutrum, leo. Pellentesque habitant morbi tristique senectus et netus et malesuada fames ac turpis egestas \cite{RFC-VP8}. Duis a eros. Class aptent taciti sociosqu ad litora torquent per conubia nostra, per inceptos hymenaeos. Fusce magna mi, porttitor quis, convallis eget, sodales ac, urna \cite{Chiang:2011fk}. \textcolor{violet}{\Cref{tab:spreadtb} illustrates the use of a Spreadsheet-like table producing calculations by columns and by lines (observe the code).} 
%
%\begin{table}[htb]
%\centering
%    \caption{A nice Spreadsheet using package ``spreadtab''. Notice the calculations.}
%    \label{tab:spreadtb}
%\begin{spreadtab}{{tabular}{rr|r}} 
%22       & 54       & a1+b1 \\
%43       & 65       & a2+b2 \\ 
%49       & 37       & a3+b3 \\
%\hline
%a1+a2+a3 & b1+b2+b3 & a4+b4
%\end{spreadtab}
%\end{table} 
% #############################################################################
\subsection{Optimization Tools in Architecture}

% #############################################################################
\subsubsection{Galapagos}

% #############################################################################
\subsubsection{Goat}
% #############################################################################
\section{Multi-Objective Optimization}
Nunc tincidunt convallis tortor. Duis eros mi, dictum vel, fringilla sit amet, fermentum id, sem. Phasellus nunc enim, faucibus ut, laoreet in, consequat id, metus. Vivamus dignissim \cite{Moscoso:2011fk}. \textcolor{violet}{\Cref{tab:comp_arch} is automatically compressed to fit text width. You can use \url{https://www.tablesgenerator.com} to produce these tables, and then copy the \LaTeX\space code generated to paste in the document.}

% #############################################################################
\subsection{Experimental Approach}

% #############################################################################
\subsection{Priori Articulation Approach}

% #############################################################################
\subsection{Pareto-Based Approach}

% #############################################################################
\subsection{Metrics for Multi-Objective Optimization}

% #############################################################################
\subsection{Optimization Tools in Architecture}

\subsubsection{Octopus}
\subsubsection{Opossum}
\subsubsection{Optimo}

%\begin{table}[h]
%\centering
%\caption{Comparison between today's and target Architectures of Telcos}
%\label{tab:comp_arch}
%\resizebox{\textwidth}{!}{%
%\begin{tabular}{|
%>{\columncolor[HTML]{ECF4FF}}l |l|
%>{\columncolor[HTML]{E2FFC9}}l |l|}
%\hline
%\multicolumn{2}{|c|}{\cellcolor[HTML]{ECF4FF}Today}                                                                                                             & \multicolumn{2}{c|}{\cellcolor[HTML]{E2FFC9}Target}                                                                                  \\ \hline
%Rigid     & \begin{tabular}[c]{@{}l@{}}Each evolutionary requirement involves \\ development of multiple components, \\ interfaces, platforms,etc.\end{tabular} & Flexible       & \begin{tabular}[c]{@{}l@{}}It is possible to modify or add new\\ functionalities rapidly.\end{tabular}              \\ \hline
%Slow      & \begin{tabular}[c]{@{}l@{}}Development of a new application takes \\ months or years.\end{tabular}                                                  & Fast           & \begin{tabular}[c]{@{}l@{}}Development of a new application takes \\ weeks instead of months or years.\end{tabular} \\ \hline
%Closed    & \begin{tabular}[c]{@{}l@{}}Limited integration with external\\ environments.\end{tabular}                                                           & Open           & \begin{tabular}[c]{@{}l@{}}It is simple to integrate internal,\\ applications with external entities.\end{tabular}  \\ \hline
%Complex   & \begin{tabular}[c]{@{}l@{}}Heterogeneous technologies, obsolescence, \\ lack,of standards, high redundancy.\end{tabular}                            & Standardised   & Use of homogeneous architectural models.                                                                            \\ \hline
%Expensive & \begin{tabular}[c]{@{}l@{}}High Capex (for new service development) \\ and,high,Opex (to ensure running of IT).\end{tabular}                        & Cost-Effective & Capex and Opex are optimised.                                                                                       \\ \hline
%\end{tabular}
%}
%\end{table}

Cras lobortis tempor velit. Phasellus nec diam ac nisl lacinia tristique. Nullam nec metus id mi dictum dignissim. Nullam quis wisi non sem lobortis condimentum. Phasellus pulvinar, nulla non aliquam eleifend, tortor wisi scelerisque felis, in sollicitudin arcu ante lacinia leo.