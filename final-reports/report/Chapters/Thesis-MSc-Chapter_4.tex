% #############################################################################
% This is Chapter 4
% !TEX root = ../main.tex
% #############################################################################
% Change the Name of the Chapter i the following line
\fancychapter{Algorithmic Optimization}
\cleardoublepage
% The following line allows to ref this chapter
\label{chap:implement}

In the last chapter, we discussed the architecture of the proposed solution, a general purpose optimization framework that is applicable to solve any optimization problem. 

From the beginning, we set out to address optimization problems involving costly functions. Therefore, during the development of the prototype of the proposed framework we focused on problems that exhibited these properties. Nevertheless, that the prototype can be easily extended to include other algorithms (e.g., derivative-based), as it has been previously discussed in ~\Cref{sec:optalgos}. 

Motivated by the large impact of the building sector in the world's sustainability and economy, this dissertation aims at applying the proposed framework to address building design optimization, thus attempting to reduce the costs and ecological footprint of buildings. Despite the existence of of multiple optimization tools in architecture (see~\Cref{sec:plugins}), these are often limited and do not provide adequate algorithms, nor mechanisms to enable the efficient optimization of problems with time-consuming evaluations (see~\Cref{sec:problemsaddress}).

In this chapter, we describe how the general-purpose framework proposed in this dissertation can be applied to address architectural design optimization. 

\section{Algorithmic Optimization}

In \Cref{ssec:ad,ssec:aa}, we have discussed how architectural paradigms have incrementally grown to develop the mechanisms to quickly (1) update a design, (2) generate the corresponding analytical model, and (3) automatically evaluate the design in an analytical tool and collect its results. 

These mechanisms laid down the foundations for automated optimization processes. By extending the Algorithmic Design (AD) and Algorithmic Analysis (AA) approaches to include optimization mechanisms, we are able to automatically apply optimization processes that aim to improve (or even optimize) a design’s performance. Using this approach, which we name \ac{AO}, architects are able to optimize their designs through (1) the creation of \ac{AD} models reflecting their design's intents, (2) the selection of the performance aspects to optimize and, thus, the analysis tools to be used (e.g., lighting, thermal, structural, costs), and, finally, (3) of the optimization algorithm. To this end, architects must specify not only the design parameters that are to be optimized and their acceptable range of values, but also which analysis to consider for optimization\footnote{\ac{AO} will treat the analysis' results for different variations of the design as the functions to optimize.}. 




As mentioned in ~\Cref{sec:problemsaddress}, the currently available architectural optimization frameworks present some limitations.

As a case study, we evaluated our proposal in the context of lighting performance. To this end, we combined different optimization methods available in existing software libraries, which allowed us to effortlessly test different algorithms. Moreover, we used the AO approach to compare the performance of ten black-box methods, cov-ering both local and global methods of the previously



- colocar algoritmos
- dar enfase a importancia de ter diferentes algoritmos.
- q algoritmos providenciados devem ter em atenção funçoes de custo pesadas...


\section{Optimization Benchmarks in Architecture}

