% #############################################################################
% This is Chapter 4
% !TEX root = ../main.tex
% #############################################################################
% Change the Name of the Chapter i the following line
\fancychapter{Evaluation}
\cleardoublepage
 
This chapter focuses on the evaluation of the proposed framework, from different perspectives, which include the diversity of the algorithms currently supported by the solution's prototype, their performance, and the visual mechanisms. To this end, we divide the evaluation in two phases: in \Cref{sec:qualitative}, we discuss the qualitative properties of the current prototype, outlining the flexibility and heterogeneity of the provided algorithms, while in \Cref{sec:quantitative}, we assess the viability of the current prototype in the architectural practice by using it to address three real case studies. These case studies use the Khepri tool and the optimizer prototype previously described, following the algorithmic framework described in \Cref{chap:implement}.

Overall, we aim to answer the following questions:
\begin{enumerate}
	\item Is our tool 
\end{enumerate}

\section{Qualitative Evaluation}
\label{sec:qualitative}

The quantitative evaluation of optimization frameworks involve considering multiple aspects, including the flexibility, adaptability, diversity of algorithms, ease of use, among others. Calling upon the \acp{NFLT} discussed in \Cref{ssec:comparisondfo}, some algorithms are really good solvers for some problems and very poor solvers for others. Selecting the right algorithm can have a great impact in the efficiency of optimization processes. Particularly, in building design, to benefit from such performance gains, diversity of algorithms allows to face each problems' characteristics differently, enabling the identification of most promising algorithms. In addition to the algorithms' diversity, in order to be easily used by less experienced users, algorithms should be effortlessly run, without the need for many manual changes. Notwithstanding their innate simplicity, optimization frameworks should also be flexible enough to enable more experienced users to fine-tune them according to their expertise, thus fostering more efficient optimization processes. At last, a good framework should be easily adaptable to handle different problems.

Regarding the adaptability of the current prototype, it provides mechanisms to address both single- and multi-objective problems: 15 \ac{SOO} algorithms and 13 \acp{MOEA}, respectively. Simpler approaches, like the design of experiments approach discussed in \Cref{ssec:doe}, are also possible using one of the 5 sampling methods available in the prototype. Moreover, the time complexity is also taken into account in this prototype, which provides a large number of \ac{ML} models to be used within model-based algorithms.

\todo{Tabela / Imagem - discriminativa sobre os tipos de algoritmos e as classes}

\todo{Table X} presents a view of the algorithms supported by the prototype discriminated by class and domain. When comparing to existing tools in architecture, our solution presents a more extense and diverse set of algorithms, which can be explored to address a wider variety of problems. Moreover, while the existing tools rely on a unique optimization approach (e.g., \ac{SOO}, or Pareto-based optimization), our solution adapts to the user needs, providing the necessary mechanisms to several approaches.

Unlike existing tools, the prototype does not present a visual \ac{GUI}, instead requiring textual programming techniques. Nevertheless, to make it more appealing to less experienced users, the prototype includes a ready-to-use format, where every algorithm is configured by default. 

- flexible - t provides enough mechanisms to allow more experienced users to use their knowledge and fine-tune the different algorithms
- benchmarks easily run - how many modifications need to be done vs grasshoppers frameworks

- visualization mechanisms?
-

- Differences / Benefits / Disadvantages when compared to Grasshopper's frameworks


% #############################################################################
\section{Quantitative of Applications}
\label{sec:quantitative}

- Dizer que de um modo geral começámos de forma incremental por considerar problemas single-objective, nomeadamente a casa da ericeira, que remonta a primeira publicação. Depois evoluimos para a avaliação bi-objetivo de dois casos de estudo reais - Pavilhão Preto para exposições e de uma arc-shaped space frame.

- Comentar a facilidade c/ que alguém que já tem um programa AD consegue acopolar optimização a AD.

% #############################################################################
\subsection{Ericeira House: Solarium}

% #############################################################################
\subsection{Black Pavilion: Arts Exhibit}

\subsubsection{Skylights Optimization}
\subsubsection{Arc-shaped Space Frame Optimization}

