% #############################################################################
% This is Chapter 6
% !TEX root = ../main.tex
% #############################################################################
% Change the Name of the Chapter i the following line
\fancychapter{Conclusion}
%\cleardoublepage
% The following line allows to ref this chapter
\label{chap:conclusion}

In this chapter, we reflect on the work that was developed and about the obtained results. We draw some general conclusions about the applicability of the solution, including in its potential for the architectural practice. We end by outlining some guidelines for future work. 

% #############################################################################
\section{Conclusions}

Nowadays, with the threat of climate change, resource depletion, and worldwide urbanization, the construction of well-designed buildings does not suffice \cite{Wortmann2015AdvSBO}. Architectural practices have, therefore, grown to incorporate considerations about the building's performance in various aspects, including energy consumption, comfort, and costs, among others. For the past few decades, the development of computational simulation tools empowered designers with the ability to simulate and obtain an estimate of a building’s performance, prior to its construction. The emergence of these tools and the raising concerns about the environmental and economic impact of buildings led to the development of new design approaches, such as Performance-Based Design (PBD), which seek more efficient design solutions by considering the designs’ performance. Taking design’s performance a step further, optimization has unveilled a new performance-based approach called Building Performance Optimization (BPO). 

Unfortunately, traditional \ac{BPO} methodologies require the evaluation of different design variations, which, in turn, imply spending a large amount of time with the manual application of changes to the design and often leading to difficulties when manually modeling complex geometry. Moreover, in order to evaluate a design's performance the corresponding analytical models must be produced, which also comprises a time-consuming and tiresome task. The emergence of algorithmic-based paradigms, like Algorithmic Design (AD) and Algorithmic (AA), enabled the implementation of automated optimization processes, as they enabled architects to generate multiple design variants with little effort, to automatically produce the corresponding analytical models, and to automatically evaluate their performance. Notwithstanding the benefits attained with algorithmic approaches, the exploration for more efficient designs still comprises a tiresome and time-consuming task. To overcome this limitation, one can exploit the automatisms associated with optimization algorithms to seek for optimal (or near optimal) design solutions more efficiently.

The development of optimization algorithms is not recent and their wide availability often difficults the selection of the most appropriate one for a specific problem. Often disregarded, the selection of the appropriate algorithm might have a significant impact in the overall efficiency of optimization processes and, also, on the quality of the retrieved results\cite{Wolpert1997NFLT}. Despite the benefits associated with a more targeted selection of optimization algorithms, most \ac{BPO} practitioners seem to only adopt \ac{EA}-based algorithms, like the ones dicussed in \cref{sec:optimizationtools}. 

As discussed in \cref{sec:optalgos}, for simulation-based optimization problems, such as \ac{BPO}, other interesting optimization algorithms can be explored, including model-based algorithms which potentially reduce the overall optimization time\cite{Wortmann2017GABESTCHOICE}. Direct-search methods have also yielded promising results in the context of \ac{BPO} problems \cite{Waibel2018}. 

% What we did, how we tested, what were the conclusions
Along these lines, we studied different optimization approaches and algorithms to measure their affinity for problems involving expensive evaluation functions, where the number of function evaluations is frequently restricted to a few hundreds \cite{Caetano2018,Belem2018optimizeddesign,Belem2019MOO}. We introduced an extension to the \ac{AD} and \ac{AA} approaches, called Algorithmic Optimization (\ac{AO}), which combines an optimization framework and an \ac{AD} tool and allows users to address various design optimization problems, including \ac{BPO}. In general, we conclude that no single class, subclass, or algorithm excels at every problem.  

Current architectural optimization tools and, in particular, \ac{MOO} tools, still rely on \ac{EA}-based algorithms. However, in the evaluated case studies, metaheuristics and, particularly, \ac{EA}-based ones were rarely the best performing algorithms. In fact, other categories, such as global model-based or global direct-search algorithms yielded better results. Different factors could change the obtained results (e.g., a different configuration for the algorithms or a lucky random step). Therefore, and contrarily to current architectural practices, we conclude that distinct algorithms behave differently according to  problems' specificity and that users should test different algorithms for a small number of evaluations or for a short amount of time. 

Furthermore, we conclude that while global optimization algorithms are quicker to converge towards optimal solutions when no additional information is known, local algorithms can be quicker if provided with good initial points and, therefore, should be considered as potential candidates when such information is available.

Regarding the evaluation of different \acp{MOOA}, the lack of consensus regarding the more appropriate way to measure their quality, makes it difficult to quantify the suitability of each algorithm for \ac{MOO} problems. In this study, we conclude that algorithms' quality should be measured through the combination of of Pareto front plots and multiple performance indicators, measuring the cardinality, diversity, and accuracy aspects of the Pareto fronts. In this way, the real impact of model-based algorithms in the overall optimization time is not quantifiable. To circumvent this limitation, one can measure these indicators per evaluation, similarly to what is done in \ac{SOO} contexts. However, this increases the complexity of the evaluation and requires more standardized ways to measure these algorithms' performance.
 
% Overall conclusion
Based on the conclusions regarding algorithmic efficiency, we believe architects should use several optimization algorithms in their workflow and use the optimization approach that better fits their needs and expertise. As previously described, optimization can be used to guide design processes towards more performative designs variants, following a \ac{BPO} methodology. Despite the existence of multiple architectural optimization plug-ins, these are often limited to a unique optimization approach and to a small subset of algorithms. In addition, each plug-in presents a distinct interface. As a result, in order to use various optimization algorithms, users must manually modify their program. In an attempt to bypass this limitation, the optimization framework proposed in this dissertation presents the necessary mechanisms to appy automated optimization processes, following an \ac{AO} approach. Moreover, it also presents mechanisms to test multiple optimization algorithms effortlessly. Besides the proposal and implementation of the framework, its suitability and capabilities were evaluated in the context of three case studies.

Overall, as discussed in this dissertation, optimization can be very beneficial for modern architecture. Unfortunately, optimization is rarely sought due to associated limitations. Despite the existence of architectural optimization tools, these are often limited, frequently leading to tiresome and to inefficient optimization processes. The generalized idea that \ac{EA}-based algorithms (e.g., \ac{GA}, \ac{SPEA2}), comprise the best approach to architectural optimization is proven to be flawed. As a matter of fact, the studies carried out in this dissertation evidence the existence of other interesting algorithms. Finally, this dissertation shows that the extending the \ac{AD} and \ac{AA} approaches to include optimization processes enables the automation of optimization processes and, particularly, of \ac{BPO} architectural practices. 
	
% #############################################################################
\section{Limitations and Future Work}

This dissertation proposes a general-purpose optimization framework providing different optimization approaches and optimization algorithms particularly tailored for optimization problems including simulation-based objective functions. Although the current framework already proved valuable to address architectural problems, it can be further improved. In this section, we describe the limitations of the current framework and we suggest future lines of research.

% System Limitations
\subsection{Limitations}

Despite providing the base functionality for addressing several optimization problems, the current solution still presents some limitations which will be emphasized in this section. When considering the key aspects mentioned in \cref{ssec:AOW}, it is possible to identify a few limitations, namely in terms of (1) the flexibility/automation of specific optimization approaches, (2) interactivity, and, also, (3) the visualization capabilities.
% In  we mentioned the four key aspects of optimization within architectural workflows. 

Firstly, notwithstanding its different optimization mechanisms to support different optimization approaches (e.g., design of experiments, \ac{SOO}, pareto-based), users could also benefit from other optimization approaches, like the prioritization of the optimization of different objectives (e.g., in a bi-objective optimization problem, the user can optimize one objective first and only then proceed to optimize the second). 

Secondly, the current framework does provide the mechanisms to follow a human-in-the-loop approach, where the user is able to interact and influence the optimization process. The ability to influence the course of an optimization approach would be particularly interesting in the architectural field, where users can use recently acquired knowledge or intuitions to explore more promising regions of the design space or even to speed-up the optimization process. A possible research path is to investigate the ability to change or add constraints in real time or to specify different design variants to be explored and generalize other similar design variants from the selected design variants.

Finally, although the current visualization mechanisms provide a fair and general understanding of the results obtained in optimization processes, these present some limitations concerning the visualization of Pareto fronts for problems involving more than three objectives. An important research path is how to represent the information regarding different runs of \ac{MOO} algorithms, for which users have a limited evaluation budget and cannot afford to let the algorithms run indefinitely until they converge towards an idyllic Pareto front. %\todo{Referenciar trabalho do Edward Tufte?} 

% One other important research path is to implement a real time implementation does not provide real-time updating mechanisms. optimization framework does not provide an integrated real-time the visual mechanisms is not being updated in real-time which is a limitation. Although we have 

\subsection{Future Work}
% Case studies
The developed framework was tested in a single-objective daylight optimization case study and two bi-objective optimization case studies, one involving the optimization of structural and aesthetics aspects and other involving the optimization of daylight and cost aspects. All three case studies involved the optimization of less than six variables, at most two objectives, and no constraints. In the future, the proposed framework could be used to test optimization problems with higher complexity, not only in terms of variables and objectives, but also by adding constraints. Moreover, it would be interesting to assess the performance of optimization algorithms when subject to optimization problems involving different design aspects, including, among others, thermal, energy consumption, and acoustics. 
 
% Optimization algorithms
One other relevant research path is the evaluation of more optimization algorithms, involving different mechanisms and strategies, in order to assess their suitability for optimization problems and its impact on the overall performance of optimization processes. Additionally, as it has been partially demonstrated with the Ericeira's case study, fine-tuning and providing initial information to certain algorithms may drastically improve their performance. For this reason, evaluating the impact of different algorithm's parameters for certain optimization problems would be another relevant case study, particularly, for complex building design problems involving expensive evaluation functions. 

Furthermore, it would be interesting to study the impact of pre-training surrogate-based optimization models based on data obtained from standard optimization test functions~\cite{Zhang2009TEST}. Particularly, users could use previous knowledge about the objectives' behavior (e.g., discontinuous, multimodal, convex) and pre-train the surrogate models in similar standardized test functions, thus, potentially, accelerating optimization processes. %An ensemble of algorithms could be exploited to maxmize the efficiency of these processes.

% - Surrogate modeling test the impact of using one surrogate model to each objective vs using one to all objectives

% Interesting future work
Finally, currently available optimization tools rarely provide any information regarding the optimization results, which difficults users' ability to understand the results. An interesting research path is to explore mechanisms that provide an explanation regarding the quality of the obtained results, especially when using opaque algorithms, i.e., whose non-linearity severely hinders the inteligibility of the optimization process itself. Recent advances in the field of \ac{ML} related to the models' explainability are promising and can be explored to provide explanations about the different optimization algorithms. 
	
	
	